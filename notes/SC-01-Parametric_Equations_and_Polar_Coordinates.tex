\documentclass[a4paper,8pt]{article}
\usepackage[a4paper, margin=15mm]{geometry}
\usepackage[T1]{fontenc}
\usepackage[utf8]{inputenc}
\usepackage{mathfunc}

% Header
% ===========================

\title{Parametric Equations and Polar Coordinates}
\author{Calculus Early Transcendentals by James Stewart (Chapter 10)}
\date{June 14\textsuperscript{th}, 2015}


% Document
% ===========================

\begin{document}
\maketitle
\pagenumbering{gobble}

\begin{outline}

  \tbullet{10.2.1 (Parametric to Cartesian)}
    If \(f'\) is continuous and \(f'(t) \neq 0\) for \(a \leq t \leq b\), then the parametric curve \(x = f(t),
    y = g(t), a \leq t \leq b\), can be put in the form \(y = F(x)\).

    \begin{proof}
      We note that if \(f'\) is continuous and \(f'(t) \neq 0\), then by the contrapositive of Fermat's Theorem,
      there are no local minimums or maximums. In other words, \(f\) must be monotonically increasing or decreasing
      and as such, must be invertible. Thus \[f^{-1}(x) = t \Rightarrow y = g(t) = g(f^{-1}(x)) = F(x)\text{.}\]
    \end{proof}

  \tbullet{10.2.2 (Derivative of Parametric Curve)}
    Let \(x = f(t)\) and \(y = g(t)\) such that the conditions in Theorem 10.2.1 are satisfied (in a piecewise
    manner if necessary). Then \(y\) can be written as \(y = F(x)\), for some function \(F\) obtained by eliminating
    the \(t\) parameter and \[ F'(x) = \frac{g'(t)}{f'(t)}\text{.} \]

    \begin{proof}
      Note \(y = F(x)\) can be written as \( g(t) = F(f(t)) \) and so, if \(f, F,\) and \(f\) are differentiable,
      the Chain Rules gives \[ g'(t) = F'(f(t))f'(t) = F'(x)f(t)\text{.} \] Thus, if \(f'(t) \neq 0\), we can
      solve for \(F'(x)\): \[ F'(x) = \frac{g'(t)}{f'(t)}\text{.} \]
    \end{proof}

  \tbullet{10.2.3 (Area of Parametric Curve)}
    Let \(x = f(t)\) and \(y = g(t)\) such that the conditions in Theorem 10.2.1 are satisfied (in a piecewise
    manner if necessary). The area under this curve is
    \[ A = \int_{\alpha}^{\beta} g(t)f'(t)\text{ or } \int_{\beta}^{\alpha} g(t)f'(t)dt\text{.} \]

    \begin{proof}
      Note \(y = F(x)\) can be written as \( g(t) = F(f(t)) \) and so, if \(f, F,\) and \(f\) are differentiable,
      the chain rule gives
      \[ A = \int_a^b ydx = \left[ \int_{\alpha}^{\beta} g(t)f'(t)dt \text{ or } \int_{\beta}^{\alpha} g(t)f'(t)dt \right] \]
      where the limits of integration
      are found also with the Substitution Rule (when \(x = a\), \(t\) is either \(\alpha\) or \(\beta\) and when
      \(x = b\), \(t\) is the remaining value).
    \end{proof}

  \tbullet{10.2.4 (Arc Length of Parametric Curve)}
    If a curve \(C\) is described by the parametric equations \(x = f(t), y = g(t), \alpha \leq t \leq \beta\), where
    \(f'\) and \(g'\) are continuous on \([\alpha, \beta]\) and \(C\) is traversed exactly once as \(t\) increases
    from \(\alpha\) to \(\beta\), then the length of \(C\) is \[ L = \int_{\alpha}^{\beta} \sqrt{[f'(t)]^2 + [g'(t)]^2}dt\text{.} \]

    \begin{proof}
      Divide the parameter interval \([\alpha, \beta]\) into \(n\) subintervals of equal width \(\Delta{t}\). If \(t_0, t_1, t_2, \ldots\)
      are endpoints of these subintervals, then \(P_i(x_i, y_i)\) is a point on \(C\) where \(x_i = f(t_i)\) and \(y_i = g(t_i)\). Then
      the polyline with vertices \(\inflatedot{P}{n}\) approximates \(C\). We define the length \(L\) to be the limit of the lengths
      of these approximating lines as \(n \rightarrow \infty\): \[ L = \xlimit{n}{\infty} \sum_{i=1}^n |P_{i-1}P_i|\text{.} \]
      The Mean Value Theorem when applied to \(f\) and \(g\) on the interval \([t_{i-1}, t_i]\) gives a number \(t_i^*\) and \(t_i^{**}\)
      in \((t_{i-1}, t_i)\) such that
      \begin{align*}
        f(t_i) - f(t_{i-1}) &= f'(t_i^*)(t_i - t_{i-1}) \text{, and } \\
        g(t_i) - g(t_{i-1}) &= g'(t_i^{**})(t_i - t_{i-1})\text{ respectively.}
      \end{align*}
      Letting \(\Delta{x_i} = x_i - x_{i-1}, \Delta{y_i} = y_i - y_{i-1}\), these equation becomes
      \[ \Delta{x_i} = f'(t_i^*)\Delta{t}\text{ and } \Delta{y_i} = g'(t_i^{**})\Delta{t}\text{.} \]
      Therefore
      \begin{align*}
        |P_{i-1}P_i| &= \sqrt{(\Delta{x_i})^2 + (\Delta{y_i})^2} = \sqrt{[f'(t_i^*)\Delta{t}]^2 + [g'(t_i^{**})\Delta{t}]^2} \\
                     &= \sqrt{[f'(t_i^*)]^2 + [g'(t_i^{**})]^2}\Delta{t}
      \end{align*}
      Despite \(t_i^* \neq t_i^{**}\) in general, if \(f'\) and \(g'\) are continous, then
      \[ L = \xlimit{n}{\infty}\sum_{i=1}^n \sqrt{[g'(t_i^*)]^2 + [g'(t_i^{**})]^2}\Delta{t} = \int_{\alpha}^{\beta} \sqrt{[f'(t)]^2 + [g'(t)]^2}dt\text{.} \]
    \end{proof}

  \tbullet{10.2.5 (Surface Area of Rotated Parametric Curve)}
    If the curve given by the parametric equations \(x = f(t), y = g(t), \alpha \leq t \leq \beta\), is rotated about the \(x\)-axis, where \(f', g'\) are
    continous and \(g(t) \geq 0\), then the area of the resulting surface is given by \[ S = \int_{\alpha}^{\beta}2\pi y\sqrt{[f'(t)]^2 + [g'(t)]^2}dt\text{.} \]

  \tbullet{10.3.1 (Tangents to Polar Curves)}
    For polar curve \(r = f(\theta)\),
    \[
      \leibniz[y]{x} = \frac{\leibniz[y]{\theta}}{\leibniz[x]{\theta}}
                     = \frac{\leibniz[r]{\theta}\sin{\theta} + r\cos{\theta}}{\leibniz[r]{\theta}\cos{\theta} - r\sin{\theta}}\text{.}
    \]

    \begin{proof}
      Regard \(\theta\) as a parameter and write its parametric equations as
      \[ x = r\cos{\theta} = f(\theta)\cos{\theta},\quad y = r\sin{\theta} = f(\theta)\sin{\theta}\text{.} \]
      Then apply parametric differentiation as in Theorem 10.2.2.
    \end{proof}

  \tbullet{10.4.1 (Area of Polar Coordinates)}
    Let \(R\) be the region bounded by the polar curve \(r = f(\theta)\) and by the rays \(\theta = a\) and \(\theta = b\), where \(f\) is a
    positive continuous function and where \(0 < b - a \leq 2\pi\). Then the area \(A\) of the polar region \(R\) is
    \[ A = \int_a^b \frac{1}{2}r^2 d\theta\text{.} \]

    \begin{proof}
      Let \(R\) be the region bounded by the polar curve \(r = f(\theta\) and by rays \(\theta = a\) and \(\theta = b\), where \(f\) is a positive continuous
      function and where \(0 < b - a \leq 2\pi\). We divide the interval \([a, b]\) into subintervals with enpoints \(\inflatedot{\theta}{n}\) and equal width
      \(\Delta{\theta}\). The rays \(\theta = \theta_i\) then divide \(R\) into \(n\) smaller regions with central angle \(\Delta{\theta} = \theta_i - \theta-{i-1}\).
      If we choose \(\theta_i^*\) in the \(\spscript{i}{th}\) subinterval \([\theta_{i-1}, \theta_i]\), then the area \(\Delta{A_i}\) of the \(\spscript{i}{th}\)
      region is approximated by the area of a sector of a circle with central angle \(\Delta{\theta}\) and radius \(f(\theta_i^*)\). Since an area of a sector
      of a circle is \(A = \frac{1}{2}r^2\theta\), we have \[\Delta{A_i} \approx \frac{1}{2}[f(\theta_i^*)]^2\Delta{\theta}\text{.}\]
      Thus the total area is \[ \xlimit{n}{\infty}\sum_{i=1}^n\frac{1}{2}[f(\theta_i^*)]^2\Delta{\theta} = \int_a^b \frac{1}{2}[f(\theta)]^2 d\theta\text{.} \]
    \end{proof}

  \tbullet{10.4.2 (Arc Length of Polar Curve)}
    Given curve \(r = f(\theta), a \leq \theta \leq b\), the length of this curve is \[ L = \int_a^b \sqrt{r^2 + [f'(\theta)]^2}d\theta\text{.} \]

    \begin{proof}
      Regard \(\theta\) as a parameter and write parametric equations of the curve as
      \[ x = r\cos{\theta} = f(\theta)\cos{\theta},\quad y = r\sin{\theta} = f(\theta)\sin{\theta}\text{.} \]
      Using the Product Rule and differentiating with respect to \(\theta\), we obtain
      \[ \leibniz[x]{\theta} = \leibniz[r]{\theta}\cos{\theta} - r\sin{\theta},\quad \leibniz[y]{\theta} = \leibniz[r]{\theta}\sin{\theta} + r\cos{\theta} \]
      so, using \(\cos^2\theta + \sin^2\theta = 1\), we have
      \[ \left(\leibniz[x]{\theta}\right)^2 + \left(\leibniz[y]{\theta}\right)^2 = \left(\leibniz[r]{\theta}\right)^2 + r^2\text{.} \]
      Assuming that \(f'\) is continuous, we then use the arc length of a parametric curve
      \[ L = \int_a^b \sqrt{\left(\leibniz[x]{\theta}\right)^2 + \left(\leibniz[y]{\theta}\right)^2}d\theta  \]
      and, substituting, find that the length of a curve with polar equation \(r = f(\theta), a \leq \theta \leq b\), is
      \[ L = \int_a^b \sqrt{r^2 + [f'(\theta)]^2}d\theta\text{.} \]
    \end{proof}

\end{outline}

\end{document}
