\documentclass[a4paper,11pt]{article}
\usepackage[a4paper, margin=20mm]{geometry}
\usepackage[T1]{fontenc}
\usepackage[utf8]{inputenc}
\usepackage{mathfunc}

% Header
% ===========================

\title{Parametric Equations and Polar Coordinates}
\author{Calculus Early Transcendentals by James Stewart (Chapter 10)}
\date{June 14\textsuperscript{th}, 2015}


% Document
% ===========================

\begin{document}
\maketitle
\pagenumbering{gobble}

\begin{outline}

  \tbullet{10.2.1 (Parametric to Cartesian)}
    If \(f'\) is continuous and \(f'(t) \neq 0\) for \(a \leq t \leq b\), then the parametric curve \(x = f(t),
    y = g(t), a \leq t \leq b\), can be put in the form \(y = F(x)\).
    
    \begin{proof}
      
    \end{proof}
    
  \tbullet{10.2.2 (Derivative of Parametric Curve)}
    Let \(x = f(t)\) and \(y = g(t)\) such that the conditions in Theorem 10.2.1 are satisfied. Then \(y\) can
    be written as \(y = F(x)\), for some function \(F\) obtained by eliminating the \(t\) parameter and 
    \[ F'(x) = \frac{g'(t)}{f'(t)}\text{.} \]
    
    \begin{proof}
      Note \(y = F(x)\) can be written as \[ g(t) = F(f(t)) \] and so, if \(f, F,\) and \(f\) are differentiable, 
      the Chain Rules gives \[ g'(t) = F'(f(t))f'(t) = F'(x)f(t)\text{.} \] Thus, if \(f'(t) \neq 0\), we can 
      solve for \(F'(x)\): \[ F'(x) = \frac{g'(t)}{f'(t)}\text{.} \]
    \end{proof}

\end{outline}

\end{document}
