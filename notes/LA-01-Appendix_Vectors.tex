\documentclass[a4paper,8pt]{article}
\usepackage[a4paper, margin=15mm]{geometry}
\usepackage[T1]{fontenc}
\usepackage[utf8]{inputenc}
\usepackage{mathfunc}

% Header
% ===========================

\title{Appendix: Vectors}
\author{Linear Algebra With Applications by Otto Bretscher (Appendix A)}
\date{June 3\textsuperscript{rd}, 2015}


% Document
% ===========================

\begin{document}
\maketitle
\pagenumbering{gobble}

\begin{outline}

  \dbullet{A.1 (Vector Addition and Scalar Multiplication)}
    \begin{enumerate}[i.]
      \item
        The sum of two vectors \(\vec{v}\) and \(\vec{w}\) in \(\bbr^n\) is defined "component-wise":
        \[
          \vec{v} + \vec{w} = \columnvec{\(v_1\), \(v_2\), \vdots, \(v_n\)} +
                              \columnvec{\(w_1\), \(w_2\), \vdots, \(w_n\)}
                            = \columnvec{\(v_1+w_1\), \(v_2+w_2\), \vdots, \(v_n+w_n\)}
        \]
      \item
        The product of a scalar \(k\) and a vector \(\vec{v}\) is defined component-wise as:
        \[
          k\vec{v} = k\columnvec{\(v_1\), \(v_2\),\vdots, \(v_n\)} = \columnvec{\(kv_1\), \(kv_2\),\vdots, \(kv_n\)}
        \]
    \end{enumerate}

  \tbullet{A.2 (Rules of Vector Algebra)}
    The following hold for all vectors \(\vec{u}, \vec{v}, \vec{w}\in\bbr^n\) and for scalars \(c\) and \(k\):
    \begin{enumerate}[i.]
      \item \((\vec{u}+\vec{v})+\vec{w} = \vec{u}+(\vec{v}+\vec{w})\)
      \item \(\vec{v}+\vec{w}=\vec{w}+\vec{v}\)
      \item \(\vec{v}+\vec{0}=\vec{v}\)
      \item For each \(\vec{v}\) in \(\bbr^n\), there exists a unique \(\vec{x}\) in \(\bbr^n\)
            such that \(\vec{v}+\vec{x} = \vec{0}\), namely, \(\vec{x} = -\vec{v}\).
      \item \(k(\vec{v}+\vec{w})=k\vec{v}+k\vec{w}\)
      \item \((c+k)\vec{v}=c\vec{v}+k\vec{v}\)
      \item \(c(k\vec{v})=(ck)\vec{v}\)
      \item \(1\vec{v}=\vec{v}\)
    \end{enumerate}

  \dbullet{A.3 (Parallel Vectors)}
    We say that two vectors \(\vec{v}\) and \(\vec{w}\) are "parallel" if one of them is a scalar multiple of the other.

  \dbullet{A.4 (Dot Product)}
    Consider two vectors \(\vec{v}\) and \(\vec{w}\) with components \(\inflatedot{v}{n}\) and \(\inflatedot{w}{n}\),
    respectively. Here \(\vec{v}\) and \(\vec{w}\) may be column or row vectors, and they need not be of the say type.
    The "dot product" of \(\vec{v}\) and \(\vec{w}\) is defined as
    \[
      \dotp{v}{w} = v_1w_1 + v_2w_2 + \cdots + v_nw_n\text{.}
    \]
    We can interpret the dot product geometrically: If \(\vec{v}\) and \(\vec{w}\) are two nonzero vectors
    in \(\bbr^n\), then
    \[
      \dotp{v}{w} = \norm{\vec{v}}\cos{\theta}\norm{\vec{w}}\text{,}
    \]
    where \(\theta\) is the angle enclosed by vectors \(\vec{v}\) and \(\vec{w}\).

  \tbullet{A.5 (Properties of Dot Products)}
    The following equations hold for all column or row vectors \(\vec{u}, \vec{v}, \vec{w}\) with \(n\)
    components, and for all scalars \(k\):
    \begin{enumerate}[i.]
      \item \(\dotp{v}{w}=\dotp{w}{v}\)
      \item \((\vec{u}+\vec{v})\cdot\vec{w} = \dotp{u}{w} + \dotp{v}{w}\)
      \item \((k\vec{v})\cdot\vec{w} = k(\dotp{v}{w})\)
      \item \(\dotp{v}{v} > 0\) for all nonzero \(\vec{v}\)
    \end{enumerate}

  \dbullet{A.6 (Norm of a Vector)}
    The "length" (or "norm") \(\norm{\vec{x}}\) of a vector \(\vec{x}\in\bbr^n\) is
    \[
      \norm{\vec{x}} = \sqrt{\dotp{x}{x}} = \sqrt{x_1^2 + x_2^2 + \cdots + x_n^2}\text{.}
    \]

  \dbullet{A.7 (Unit Vector)}
    A vector \(\vec{u}\) in \(\bbr^n\) is called a "unit vector" if \(\norm{\vec{u}} = 1\); that is,
    its length is \(1\).

  \dbullet{A.8 (Perpendicular Vectors)}
    Two vectors \(\vec{v}\) and \(\vec{w}\) in \(\bbr^n\) are called "perpendicular" (or "orthogonal")
    if \(\dotp{v}{w} = 0\).

  \dbullet{A.9 (Cross Product in \(\bbr^3\))}
    The cross product \(\crossp{v}{w}\) of two vectors \(\vec{v}\) and \(\vec{w}\) in \(\bbr^3\) is the vector
    in \(\bbr^3\) with the following three properties:
    \begin{enumerate}[i.]
      \item
        \(\crossp{v}{w}\) is orthogonal to both \(\vec{v}\) and \(\vec{w}\).
      \item
        \(\norm{\crossp{v}{w}} = \norm{v}\sin{\theta}\norm{w}\), where \(\theta\) is the angle between
        \(\vec{v}\) and \(\vec{w}\), with \(0 \leq \theta \leq \pi\). This means that the magnitude of the
        vector \(\crossp{v}{w}\) is the area of the parallelogram spanned by \(\vec{v}\) and \(\vec{w}\).
      \item
        The direction of \(\crossp{v}{w}\) follows the "right-hand rule."
    \end{enumerate}

  \tbullet{A.10 (Properties of the Cross Product)}
    The following equations hold for all vectors \(\vec{u}, \vec{v}, \vec{w} \in \bbr^3\) and for all scalars \(k\):
    \begin{enumerate}[i.]
      \item \(\crossp{w}{v} = -(\crossp{v}{w})\)
      \item \((k\vec{v})\times\vec{w} = k(\crossp{v}{w}) = \vec{v}\times(k\vec{w})\)
      \item \(\vec{v}\times(\vec{u}+\vec{w})=\crossp{v}{u}+\crossp{v}{w}\)
      \item \(\crossp{v}{w}=\vec{0}\) if and only if \(\vec{v}\) is parallel to \(\vec{w}\)
      \item \(\crossp{v}{v} = \vec{0}\)
      \item \(\vec{e}_1\times\vec{e}_2=\vec{e}_3,\vec{e}_2\times\vec{e}_3=\vec{e}_1,\vec{e}_3\times\vec{e}_1=\vec{e}_2\)
    \end{enumerate}

\end{outline}

\end{document}
