\documentclass[a4paper,11pt]{article}
\usepackage[a4paper, margin=20mm]{geometry}
\usepackage[T1]{fontenc}
\usepackage[utf8]{inputenc}
\usepackage{mathfunc}
\usepackage{minted}

% Header
% ===========================

\title{Functional Laws (Functors, Applicatives, Monoids, Monads)}
\author{Learn You a Haskell for Great Good by Miran Lipovaca (Chapters 11, 12, 13)}
\date{June 8\textsuperscript{th}, 2015}


% Document
% ===========================

\begin{document}
\maketitle
\pagenumbering{gobble}

\begin{outline}

  \tbullet{11.1 (Functor Laws)}
    For functors
    \begin{figure}[thp]
      \centering 
      \begin{minipage}{0.4\textwidth}
        \begin{minted}{haskell}
class Functor f where
    fmap :: (a -> b) -> f a -> f b
        \end{minted}
      \end{minipage}
    \end{figure}
    
    the following laws hold:
    \begin{enumerate}[i.]
      \item 
        \begin{minted}{haskell}
  fmap id = id
        \end{minted}
      \item
        \begin{minted}{haskell}
  fmap (f . g) = fmap f . fmap g
        \end{minted}
    \end{enumerate}
    
  \tbullet{11.2 (Applicative Laws)} 
    For applicative functors
    \begin{figure}[thp]
      \centering 
      \begin{minipage}{0.4\textwidth}
        \begin{minted}{haskell}
class Functor f => Applicative f where
    pure  :: a -> f a
    (<*>) :: f (a -> b) -> f a -> f b
    (*>)  :: f a -> f b -> f b
    (<*)  :: f a -> f b -> f a
        \end{minted}
      \end{minipage}
    \end{figure}
    
    the following laws hold:
    \begin{enumerate}[i.]
      \item
        \begin{minted}{haskell}
  pure id <*> v = v
        \end{minted}
      \item
        \begin{minted}{haskell}
  pure (.) <*> u <*> v <*> w = u <*> (v <*> w)
        \end{minted}
      \item
        \begin{minted}{haskell}
  pure f <*> pure x = pure (f x)
        \end{minted}
      \item
        \begin{minted}{haskell}
  u <*> pure y = pure ($ y) <*> u
        \end{minted}
    \end{enumerate}
  
  \tbullet{12.1 (Monoid Laws)}
    For monoids
    \begin{figure}[thp]
      \centering 
      \begin{minipage}{0.4\textwidth}
        \begin{minted}{haskell}
class Monoid a where
    mempty  :: a
    mappend :: a -> a -> a
    mconcat :: [a] -> a
        \end{minted}
      \end{minipage}
    \end{figure}
    
    the following laws hold:
    \begin{enumerate}[i.]
      \item
        \begin{minted}{haskell}
  mempty `mappend` x = x
        \end{minted}
      \item
        \begin{minted}{haskell}
  x `mappend` mempty = x
        \end{minted}
      \item
        \begin{minted}{haskell}
  (x `mappend` y) `mappend` z = x `mappend` (y `mappend` z)
        \end{minted}
    \end{enumerate}
    
  \pagebreak
  \tbullet{13.1 (Monad Laws)}
    For monads
    \begin{figure}[thp]
      \centering 
      \begin{minipage}{0.4\textwidth}
        \begin{minted}{haskell}
class Monad m where
    (>>=)  :: m a -> (a -> m b) -> m b
    (>>)   :: m a -> m b -> m b
    return :: a -> m a
    fail   :: String -> m a
        \end{minted}
      \end{minipage}
    \end{figure}
    
    the following laws hold:
    \begin{enumerate}[i.]
      \item
        \begin{minted}{haskell}
  return x >>= f = f x
        \end{minted}
      \item
        \begin{minted}{haskell}
  m >>= return = m
        \end{minted}
      \item
        \begin{minted}{haskell}
  (m >>= f) >>= g = m >>= (\x -> f x >>= g)
        \end{minted}
    \end{enumerate}
    
\end{outline}

\end{document}
