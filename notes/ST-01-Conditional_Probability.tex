\documentclass[a4paper,8pt]{article}
\usepackage[a4paper, margin=15mm]{geometry}
\usepackage[T1]{fontenc}
\usepackage[utf8]{inputenc}
\usepackage{mathfunc}

% Header
% ===========================

\title{Conditional Probability}
\author{A First Course in Probability by Sheldon Ross (Chapter 3)}
\date{May 29\textsuperscript{th}, 2015}


% Document
% ===========================

\begin{document}
\maketitle
\pagenumbering{gobble}

\begin{outline}

  \dbullet{3.2.1}
    If \(\bbp[F] > 0\), then \(\bbp[E|F] = \frac{\bbp[EF]}{\bbp[F]}\).

  \tbullet{3.2.2}
    \(\bbp\{\inflatedot[]{E}{n}\} = \bbp[E_1]\bbp[E_2|E_1]\ldots \bbp\{E_n|\inflatedot[]{E}{n-1}\}\).

  \tbullet{3.3.1}
    \(\bbp[E] = \bbp[E|F]\bbp[F] + \bbp[E|F^c]\bbp[F^c]\)

    \begin{proof}
      Note the following:
      \begin{align*}
        \bbp[E] &= \bbp[EF \cup EF^c]    \\
                &= \bbp[EF] + \bbp[EF^c] \\
                &= \bbp[E|F]\bbp[F]      \\
                &= \bbp[E|F^c]\bbp[F^c]
      \end{align*}
    \end{proof}

  \tbullet{3.3.2}
    The "odds" of an event \(A\) are defined by \(\frac{\bbp[A]}{\bbp[A^c]}\). That is,
    how much more likely it is event \(A\) occurs than it is that it doesn't.

  \pbullet{3.3.3}
    The new odds after evidence \(E\) is introduced are
    \[
      \frac{\bbp[H|E]}{\bbp[H^c|E]} =
      \frac{\bbp[H]\bbp[E|H]}{\bbp[H^c]\bbp[E|H^c]}
      \text{.}
    \]

    \begin{proof}
      Note the following:
      \begin{align*}
        \bbp[H|E]   &= \frac{\bbp[E|H]\bbp[H]}{\bbp[E]}\\
        \bbp[H^c|E] &= \frac{\bbp[E|H^c]\bbp[H^c]}{\bbp[E]}
      \end{align*}
      Divide two equations above for answer.
    \end{proof}

  \pbullet{3.3.4}
    Suppose \(\inflatedot{F}{n}\) are mutually exclusive events such that \(\bigcup_{i=1}^n F_i = S\). Then
    \(\bbp[E] = \sum_{i=1}^n \bbp[E|F_i]\bbp[F_i]\).

  \tbullet{3.3.5 (Bayes's Theorem)}
    Let \(\inflatedot{F}{n}\) be mutually exclusive, exhaustive events. Then given event \(E\),
    \begin{align*}
      \bbp[F_j|E] &= \frac{\bbp[EF_j]}{\bbp[E]} \\
                        &= \frac{\bbp[E|F_j]\bbp[F_j]}{\sum_{i=1}^n \bbp[E|F_i]\bbp[F_i]}
                           \text{.}
    \end{align*}

  \dbullet{3.4.1}
    Two events \(E\) and \(F\) are said to be "independent" if \(\bbp[EF] = \bbp[E]\bbp[F]\).
    Two events that are not independent are said to be dependent.

  \pbullet{3.4.2}
    If \(E\) and \(F\) are independent, then so are \(E\) and \(F^c\).

    \begin{proof}
      Assume \(E\) and \(F\) are independent. Since
      \begin{align*}
                    E &= EF \cup EF^c \Rightarrow   \\
              \bbp[E] &=\bbp[EF] + \bbp[EF^c]       \\
                      &= \bbp[E]\bbp[F] + \bbp[EF^c]
      \end{align*}
      Thus
      \begin{align*}
        \bbp[EF^c] &= \bbp[E] - \bbp[E]\bbp[F]\\
                         &= \bbp[E][1-\bbp[F]]\\
                         &= \bbp[E]\bbp[F^c]
      \end{align*}
    \end{proof}

  \dbullet{3.4.3}
    Three events \(E, F,\) and \(G\) are said to be independent if
    \begin{align*}
      \bbp[EFG] &= \bbp[E]\bbp[F]\bbp[G],       \\
      \bbp[EF]  &= \bbp[E]\bbp[F],              \\
      \bbp[EG]  &= \bbp[E]\bbp[G], \text{ and } \\
      \bbp[FG]  &= \bbp[F]\bbp[G]
    \end{align*}

  \dbullet{3.4.4}
    The events \(\inflatedot{E}{n}\) are said to be independent if for every subset \(E_{1'}, E_{2'}, \ldots,
    E_{r'}\), \(r \leq n\), of these events, \(\bbp[E_{1'}E_{2'}\ldots E_{r'}] = \bbp[E_{1'}]\bbp[E_{2'}]
    \ldots\bbp[E_{r'}]\). An infinite set of events are independent if every finite subset of those events
    is independent.

  \pbullet{3.5.1}
    Conditional probabilities satisfy the axioms of probability:
    \begin{enumerate}[i.]
      \item \(0 \leq \bbp[E|F] \leq 1\)
      \item \(\bbp[S|F] = 1\)
      \item If \(E_i, i = 1, \ldots\), are mutually exclusive events, then \(\bbp(\bigcup_{1}^{\infty}E_i|F) =
      \sum_{1}^{\infty}\bbp[E_i|F]\).
    \end{enumerate}

    \begin{proof}
      \begin{enumerate}[i.]
        \item
          Note \(0 \leq \bbp[E|F]\) is obvious. Now \(\bbp[E|F] = \frac{\bbp[EF]}{\bbp[F]}\) and note
          \(EF \subset F\). Thus \(\bbp[EF] \leq \bbp[F]\).
        \item
          \(\bbp[S|F] = \frac{\bbp[SF]}{\bbp[F]} = \frac{\bbp[F]}{\bbp[F]} = 1\).
        \item
          \(\bbp[\bigcup_{1}^{\infty} E_i|F] = \frac{\bbp\{\left[\bigcup_{1}^{\infty} E_i\right]F\}}{\bbp[F]}
          = \frac{\bbp[\bigcup_{1}^{\infty}E_iF]}{\bbp[F]}\). This in turn is
          \(\frac{\sum_{1}^{\infty} \bbp[E_iF]}{\bbp[F]} = \sum\limits_{1}^{\infty} \frac{\bbp[E_iF]}{\bbp[F]}\).
      \end{enumerate}
    \end{proof}
\end{outline}

\end{document}
