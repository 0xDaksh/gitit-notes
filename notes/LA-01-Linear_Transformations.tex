\documentclass[a4paper,8pt]{article}
\usepackage[a4paper, margin=15mm]{geometry}
\usepackage[T1]{fontenc}
\usepackage[utf8]{inputenc}
\usepackage{mathfunc}

% Header
% ===========================

\title{Linear Transformations}
\author{Linear Algebra With Applications by Otto Bretscher (Chapter 2)}
\date{May 29\textsuperscript{th}, 2015}


% Document
% ===========================

\begin{document}
\maketitle
\pagenumbering{gobble}

\begin{outline}

  % Section 2.1
  % ----------------------------------------------------------------------------

  \dbullet{2.1.1 (Linear Transformations)}
    A function \(T\) from \(\bbr^m\) to \(\bbr^n\) is called a "linear transformation" if there exists
    an \(n \times m\) matrix \(A\) such that \(T(\vec{x}) = A\vec{x}\), for all \(\vec{x} \in \bbr^m\).

  \tbullet{2.1.2 (Properties of Linearity)}
    Consider a linear transformation from \(T\) from \(\bbr^m\) to \(\bbr^n]\). Then the matrix of \(T\) is:
    \[
      A =
        \begin{bmatrix}
          \vert  & \vert  &        & \vert  \\
          T(e_1) & T(e_2) & \ldots & T(e_m) \\
          \vert  & \vert  &        & \vert
        \end{bmatrix}
      ,\text{ where } e_{i} = \begin{bmatrix} 0 \\ \vdots \\ 1 \\ \vdots \\ 0 \end{bmatrix}
    \]
    and \(1\) takes the \(i^{\text{th}}\) component of \(\vec{e}_i\).

    \begin{proof}
      Consider linear transformation \(T\) such that \(T(\vec{x}) = A\vec{x}\) for matrix \(A\). Note that
      \(T(\vec{e}_1)\) is the first column of A, \(T(\vec{e}_2)\) is the second column, etc. In general, the
      \(i^{\text{th}}\) column of A is \(T(\vec{e}_i)\). The theorem then follows.
    \end{proof}

  \tbullet{2.1.3}
    A transformation \(T\) from \(\bbr^m\) to \(\bbr^n\) is linear iff:
    \begin{enumerate}[i.]
      \item \(T(\vec{v} + \vec{w}) = T(\vec{v}) + T(\vec{w})\), for all \(\vec{v}, \vec{w} \in \bbr^m\), and
      \item \(T(k\vec{v}) = kT(\vec{v})\), for all \(\vec{v} \in \bbr^m\) and all scalars \(k\).
    \end{enumerate}

    \begin{proof}
      \forward
        First, suppose \(T\) is a linear transformation. Then \(T(\vec{v} + \vec{w}) = A(\vec{v} + \vec{w})
        = A\vec{v} + a\vec{w} = T(\vec{v}) + T(\vec{w})\). Also, \(T(k\vec{v}) = A(k\vec{v}) = kA(\vec{v})
        = kT(\vec{v})\).

      \backward
        Now, suppose \(T: \bbr^m \rightarrow \bbr^n\) is a transformation that satisfied
        (\romannumeral 1) and (\romannumeral 2). We must show a matrix \(A\) exists such that
        \(T(\vec{x}) = A\vec{x}\), for all \(\vec{x} \in \bbr^m\). Let \(\vec{e_1}, \vec{e_2},
        \ldots, \vec{e_m}\) be the standard vectors. Then
        \begin{align*}
          T(\vec{x}) &= T(x_1\vec{e}_1 + x_2\vec{e}_2 + \ldots + x_m\vec{e}_m)\\
                     &= T(x_1\vec{e}_1) + T(x_2\vec{e}_2) + \ldots + T(x_m\vec{e}_m)
                        \text{ by property (\romannumeral 1)}\\
                     &= x_1T(\vec{e}_1) + x_2T(\vec{e}_2) + \ldots + x_mT(\vec{e}_m)
                        \text{ by property (\romannumeral 2).}
        \end{align*}
        Thus
        \[ A = \begin{bmatrix}
          \vert        & \vert        &        & \vert       \\
          T(\vec{e}_1) & T(\vec{e}_2) & \ldots & T(\vec{e}_m)\\
          \vert        & \vert        &        & \vert       \end{bmatrix}],
        \] proving a matrix exists.
    \end{proof}


  % Section 2.2
  % ----------------------------------------------------------------------------

  \dbullet{2.2.1 (Scaling)}
    For any positive constant \(k\), the matrix \(\begin{bsmallmatrix} k&0\\ 0&k \end{bsmallmatrix}\) defines
    a scaling by \(k\). This is a "dilation" for \(k > 1\) and a "contraction" for \(0 < k < 1\).

  \dbullet{2.2.2 (Orthogonal Projection)}
    The transformation \(T(\vec{x}) = \vec{x}^{\parallel}\) from \(\bbr^2\) to \(\bbr^2\)
    is called the "orthogonal projection of \(\vec{x}\) onto \(L\), often denoted by \(\text{proj}_{L}(\vec{x})\).

  \tbullet{2.2.3}
    \(\text{proj}_{L}(\vec{x}) = (\frac{\dotp{x}{w}}{\dotp{w}{w}})\vec{w}\)
    where \(\vec{w}\) is a nonzero vector parallel to \(L\), with matrix \(\frac{1}{w_1^2 + w_2^2}
    \begin{bsmallmatrix} w_1^2&w_1w_2\\ w_1w_2&w_2^2 \end{bsmallmatrix}\).

    \begin{proof}
      Let \(\vec{w}\) be a nonzero vector parallel to \(L\). Since \(\vec{x}^{\parallel}\)
      is parallel to \(\vec{w}\),
      \(\vec{x}^{\parallel} = k\vec{w}\) for some scalar \(k\). Now:
      \[
        \vec{x}^{\perp} = \vec{x} - \vec{x}^{\parallel} = \vec{x} - k\vec{w}
      \]
      is perpendicular to \(L\). Thus
      \begin{align*}
        (\vec{x} - k\vec{w}) \cdot \vec{w} &= 0 \Rightarrow \dotp{x}{w} - k(\dotp{w}{w})\\
                                           &= 0 \Rightarrow k = \frac{\dotp{x}{w}}{\dotp{w}{w}}
      \end{align*}
    \end{proof}

  \dbullet{2.2.4 (Reflection)}
    Consider a line \(L\) in the coordinate plane, running through the origin, and let
    \(\vec{x} = \vec{x}^{\parallel} + \vec{x}^{\perp}\) be a vector in \(\bbr^2\). The linear transformation
    \(T(\vec{x}) = \vec{x}^{\parallel} - \vec{x}^{\perp}\) is called the "reflection of \(\vec{x}\) about \(L\),"
    denoted \(\text{ref}_{L}(\vec{x})\).

  \tbullet{2.2.5}
    \(\text{ref}_{L}(\vec{x}) = \text{proj}_{L}(\vec{x}) - \vec{x}\), yielding matrix of form
    \(\begin{bsmallmatrix} a&b\\ b&-a \end{bsmallmatrix}\) where \(a^2 + b^2 = 1\), if working with unit vectors.

    \begin{proof}
      Note the following:
      \begin{align*}
        \text{ref}_{L}(\vec{x}) &= \vec{x}^{\parallel} - \vec{x}^{\perp}\\
                                &= \vec{x}^{\parallel} - (\vec{x} - \vec{x}^{\parallel})\\
                                &= 2\vec{x}^{\parallel} - \vec{x}\\
                                &= 2\text{proj}_{L}(\vec{x})- \vec{x}\\
                                &= (\frac{\dotp{x}{w}}{\dotp{w}{w}})\vec{w} - \vec{x}\\
                                &= \frac{x_1w_1 + x_2w_2}{w_1^2 + w_2^2} \begin{bmatrix} w_1 \\ w_2
                                   \end{bmatrix}-\begin{bmatrix} x_1 \\ x_2 \end{bmatrix}\\
                                &= \frac{1}{w_1^2 + w_2^2}
                                   \begin{bmatrix}
                                     (w_1^2 - w_2^2) & 2w_1w_2 \\
                                     2w_1w_2 & (w_2^2 - w_1^2)
                                   \end{bmatrix}
                                   \begin{bmatrix} x_1 \\ x_2 \end{bmatrix}
      \end{align*}
    \end{proof}

  \tbullet{2.2.6 (Rotation)}
    The matrix of a counterclockwise rotation in \(\bbr^2\) through an angle \(\theta\) is
    \(\begin{bsmallmatrix} \cos{\theta} & -\sin{\theta} \\ \sin{\theta} & \cos{\theta} \end{bsmallmatrix}\)
    where this matrix is of form \(\begin{bsmallmatrix} a & -b \\ b & a \end{bsmallmatrix}\) and \(a^2 + b^2 = 1\).

    \begin{proof}
      Consider
      \[
        \vec{x} = \begin{bmatrix} x_1 \\ x_2 \end{bmatrix} \text{ and }
        \vec{y} = \begin{bmatrix} y_1 \\ y_2 \end{bmatrix} \text{,}
      \]
      obtained by rotating \(\vec{x}\) by \(\frac{\pi}{2}\). We see \(T(\vec{x})\),
      the transformation that rotates \(\vec{x}\) by \(\theta\), is \((\cos{\theta})\vec{x}
      + (\sin{\theta})\vec{y}\). Thus:
      \begin{align*}
        T(\vec{x}) &= \cos{\theta} \begin{bmatrix}  x_1 \\ x_2 \end{bmatrix} +
                      \sin{\theta} \begin{bmatrix} -x_2 \\ x_1 \end{bmatrix} \\
                   &= \begin{bmatrix}
                        \cos{\theta} & -\sin{\theta}           \\
                        \sin{\theta} &  \cos{\theta}
                      \end{bmatrix}\text{,}
      \end{align*}
      and we note \(cos^2\theta + sin^2\theta = 1\).
    \end{proof}

  \tbullet{2.2.7}
    A matrix of the form \(\begin{bsmallmatrix} a & -b \\ b & a \end{bsmallmatrix}\) represents a rotation combined
    with a scaling. More precisely, if \(r\) and \(\theta\) are polar coordinates of \(\begin{bsmallmatrix}
    a \\ b \end{bsmallmatrix}\), then \(\begin{bsmallmatrix} a & -b \\ b & a \end{bsmallmatrix}\) represents a rotation
    \(\theta\) with a scaling by \(r\).

  \tbullet{2.2.8 (Shearing)}
    The matrix of a "horizontal shear" is of the form \(\begin{bsmallmatrix} 1 & k \\ 0 & 1 \end{bsmallmatrix}\), and the
    matrix of a "vertical shear" is of the form \(\begin{bsmallmatrix} 1 & 0 \\ k & 1 \end{bsmallmatrix}\), where \(k\) is
    and arbitrary constant.

  % Section 2.3
  % ----------------------------------------------------------------------------

  \tbullet{2.3.1 (Linear Composition)}
    Let \(A\) be a \(p \times n\) matrix and \(B\) be a \(m \times p\) matrix. Then the transformation
    \(\vec{z} = B(A\vec{x})\) is linear.

    \begin{proof}
      Let \(T(\vec{x}) = B(A\vec{x})\). Then
      \begin{align*}
        T(\vec{v}+\vec{w}) &= B(A(\vec{v}+\vec{w}))     \\
                           &= B(A\vec{v}+A\vec{w})      \\
                           &= B(A\vec{v}) + B(A\vec{w})
      \end{align*}
      Similarly, \(T(k\vec{v}) = B(A(k\vec{v})) = B(kA\vec{v}) = kB(A\vec{v})\).
    \end{proof}

  \dbullet{2.3.2 (Matrix Product)}
    The matrix of the linear transformation \(\vec{z} = B(A\vec{x})\) is called the "product" of the matrices \(B\)
    and \(A\), written as \(BA\).

  \tbullet{2.3.3}
    Let \(B\) be an \(n \times p\) matrix and \(A\) a \(q \times m\) matrix. The product \(BA\) is defined
    if and only if \(p = q\).

  \tbullet{2.3.4}
    If \(B\) is an \(n \times p\) matrix and \(A\) a \(p \times m\) matrix, then the product \(BA\) is
    an \(n \times m\) matrix.

  \tbullet{2.3.5}
    Let \(B\) be an \(n \times p\) matrix and \(A\) a \(p \times m\) matrix columns with columns
    \(\vec{v}_1, \vec{v}_2, \ldots, \vec{v}_m\). Then \(BA = B\begin{bmatrix} \vec{v}_1 & \vec{v}_2 & \ldots
    & \vec{v}_m\end{bmatrix} = \begin{bmatrix} B\vec{v}_1 & B\vec{v}_2 & \ldots & B\vec{v}_m\end{bmatrix}\).

    \begin{proof}
      The \(\spscript{i}{th} \text{ column of } BA) = (BA)\vec{e}_i = B(A\vec{e}_i) = B(\spscript{i}{th}
      \text{ column of } A)\).
    \end{proof}

  \tbullet{2.3.6}
    \(AB \neq BA\), in general.

  \dbullet{2.3.7 (Matrix Commutativity)}
    If \(AB = BA\), we say the matrices \(A\) and \(B\) "commute".

  \tbullet{2.3.8}
    Let \(B\) be an \(n \times p\) matrix and \(A\) a \(p \times m\) matrix. The \(\spscript{ij}{th}\) entry
    of \(BA\) is the dot product of the \(\spscript{i}{th}\) row of \(B\) with the \(\spscript{j}{th}\) column
    of \(A\). That is, \(BA\) is the \(n \times m\) matrix whose \(\spscript{ij}{th}\) entry is \(b_{i1}a_{1j} +
    b_{i2}a_{2j} + \ldots + b_{ip}a_{pj} = \sum_{k=1}^p b_{ik}a_{kj}\).

  \tbullet{2.3.9}
    For an \(n \times m\) matrix \(A\), \(AI_m = I_nA = A\).

  \tbullet{2.3.10}
    \((AB)C = A(BC)\). We simply write \(ABC\) in this case.

  \tbullet{2.3.11}
    Let \(A, B\) be \(n \times p\) matrices, and \(C,D\) be \(p \times m\) matrices. Then \(A(C + D) = AC + AD\),
    and \((A + B)C = AC + BC\).

  \tbullet{2.3.12}
    If \(A\) is an \(n \times p\) matrix, \(B\) is a \(p \times m\) matrix, and \(k\) is a scalar, then \((kA)B =
    A(kB) = k(AB)\).

  \tbullet{2.3.13}
    Block matrices can be multiplied as though the blocks were scalars (i.e. like in Theorem 2.3.6) provided
    that all submatrix products are defined.

  \dbullet{2.4.1 (Function Inverses)}
    A function \(T\) from \(X\) to \(Y\) is called "invertible" if the equation \(T(x) = Y\) has a unique solution
    \(x \in X\) for each \(y \in Y\).

  \dbullet{2.4.2 (Matrix Invertibility)}
    A square matrix \(A\) is said to be "invertible" if the linear transformation \(\vec{y} = T(\vec{x}) = A\vec{x}\)
    is invertible. In this case, the matrix of \(T^{-1}\) is denoted by \(A^{-1}\). If the linear transformation
    \(\vec{y} = T(\vec{x}) = A\vec{x}\) is invertible, then its inverse \(\vec{x} = T^{-1}(\vec{y}) = Ay^{-1}\).

  \tbullet{2.4.3}
    An \(n \times n\) matrix \(A\) is invertible if and only if rref\((A) = I_n\) or, equivalently, if
    rank\((A) = n\).

  \tbullet{2.4.4}
    Let \(A\) be an \(n \times n\) matrix. If \(A\) is invertible, then the system \(A\vec{x}=\vec{b}\) has
    the unique solution \(\vec{x} = A^{-1}\vec{b}\). Else, the system \(A\vec{x} = \vec{b}\) has infinitely
    many solutions or none.

  \tbullet{2.4.5}
    Let \(A\) be an \(n \times n\) matrix. Let \(\vec{b} \in \bbr^n\) such that \(\vec{b} = \vec{0}\).
    The system \(A\vec{x} - \vec{b}\) has \(\vec{x} = \vec{0}\) as a solution. If \(A\) is invertible, then
    this is the only solution. Otherwise \(A\vec{x} = \vec{b}\) has infinitely many solutions.

  \tbullet{2.4.6}
    For an invertible \(n \times n\) matrix \(A\), \(A^{-1}A = I_n\) and \(AA^{-1} = I_n\).

  \tbullet{2.4.7 (Invertible Composition)}
    If \(A\) and \(B\) are invertible \(n \times n\) matrices, then \(BA\) is also invertible, and
    \((BA)^{-1} = A^{-1}B^{-1}\).

    \begin{proof}
      Let \(A,B\) be invertible matrices.
      \begin{align*}
        \vec{y} &= BA\vec{x}                                 \\
                &\Rightarrow B^{-1}\vec{y} = B^{-1}BA\vec{x} \\
                &\Rightarrow B^{-1}\vec{y} = A\vec{x}        \\
                &\Rightarrow A^{-1}B^{-1}\vec{y} = \vec{x}
      \end{align*}
    \end{proof}

  \tbullet{2.4.8 (Determinant)}
    The \(2 \times 2\) matrix \(A = \begin{bsmallmatrix} a & b \\ c & d \end{bsmallmatrix}\) is invertible if and only
    if \(ad-bc \neq 0\). Quantity \(ad - bc\) is called the "determinant" of \(A\), written \(\text{det}(A)\).
    If A is invertible, then \(A^{-1} = \frac{1}{\text{det}(A)} \begin{bsmallmatrix} d & -b \\ -c & a \end{bsmallmatrix}\).

  \tbullet{2.4.9 (Geometric Interpretation of Determinant)}
    If \(A = \begin{bmatrix} \vec{v} & \vec{w} \end{bmatrix}\) is a \(2 \times 2\) matrix with nonzero columns
    \(\vec{v}\) and \(\vec{w}\), then \(\text{det}(A) = \text{det}\begin{bmatrix} \vec{v} & \vec{w} \end{bmatrix} =
    |\vec{v}|\sin{\theta}|\vec{w}|\), where \(\theta\) is the oriented angle from \(\vec{v}\) to \(\vec{w}\),
    with \(-\pi < \theta \leq \pi\).

    \begin{proof}
      Let \(A = \begin{bmatrix} \vec{v} & \vec{w} \end{bmatrix}\) be a \(2 \times 2\) matrix with
      \[
        \vec{v} = \begin{bmatrix} a \\ c \end{bmatrix},
        \vec{w} = \begin{bmatrix} b \\ d \end{bmatrix}, \text{ and }
        \vec{x} = \begin{bmatrix} -c \\ a \end{bmatrix},
      \]
      where \(\vec{x}\) is the auxiliary vector obtained by a rotation of \(\vec{v}\) by \(\sfrac{\pi}{2}\) radians.
      Let \(\theta\) be the oriented angle from \(\vec{v}\) to \(\vec{w}\), with \(-\pi < \theta \leq \sfrac{\pi}{2}\).
      Then \(\text{det}(A) = ad-bc = \dotp{x}{w} = |\vec{x}|\cos{(\frac{\pi}{2} \cdot \theta)}|\vec{w}|
      = |\vec{v}|\sin{\theta}|\vec{w}|\).
    \end{proof}

  \cbullet{2.4.10}
    \begin{enumerate}[i.]
      \item
        \(|\text{det}(A)|=|\vec{v}||\sin{\theta}||\vec{w}|\) is the area of the parallelogram spanned by
        \(\vec{v}\) and \(\vec{w}\).
      \item
        \(\text{det}(A) = 0\) if \(\vec{v}\) and \(\vec{w}\) are parallel, meaning \(\theta = 0\) or
        \(\theta = \pi\).
      \item
        \(\text{det}(A) > 0\) if \(0 < \theta < \pi\) and \(\text{det}(A) < 0\) if \(-\pi < \theta < 0\).
    \end{enumerate}

\end{outline}

\end{document}
