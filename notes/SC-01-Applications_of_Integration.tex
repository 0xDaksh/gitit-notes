\documentclass[a4paper,11pt]{article}
\usepackage[a4paper, margin=20mm]{geometry}
\usepackage[T1]{fontenc}
\usepackage[utf8]{inputenc}
\usepackage{mathfunc}

% Header
% ===========================

\title{Applications of Integration}
\author{Calculus Early Transcendentals by James Stewart (Chapter 6)}
\date{June 2\textsuperscript{nd}, 2015}


% Document
% ===========================

\begin{document}
\maketitle
\pagenumbering{gobble}

\begin{outline}

  \tbullet{6.1.2}
    The area \(A\) of the region bounded by the curves \(y = f(x)\) and \(y = g(x)\) and between \(x=a\) and \(x=b\),
    where \(b > a \geq 0\), is \[ A = \int_a^b |f(x)-g(x)|dx\text{.} \]
    
  \dbullet{6.2.1}
    Let \(S\) be a solid that lies between \(x = a\) and \(x = b\). If the cross-sectional area of \(S\) in the
    plane \(P_x\), through \(x\) and perpendicular to the \(x\)-axis, is \(A(x)\), where \(A\) is a continuous function,
    then the "volume" of \(S\) is \[V = \xlimit{n}{\infty} \sum_{i=1}^n A(x_i^*)\Delta x = \int_a^bA(x)dx\text{.}\]
    
  \tbullet{6.3.2 (Volumes By Cylindrical Shells)}
    Let \(S\) be a solid obtained by rotating about the \(y\)-axis the region bounded by \(y = f(x)\) (where \(f(x)
    \geq \)), \(y = 0, x = a\), and \(x = b\), where \(b > a \geq 0\). Then the volume of \(S\) is 
    \[ V= \int_a^b 2\pi x f(x)dx\text{, where } 0 \leq a < b\text{.} \]
    
    \begin{proof}
    Consider a cylindrical shell with inner radius \(r_1\), outer radius \(r_2\), and height \(h\). Its volume \(V\)
    can be calculated by subtracting the volume \(V_1\) of the inner cyclinder from the volume \(V_2\) of the
    outer cylinder:
    \begin{align*}
      V &= V_2 - V_1 \\
        &= \pi r_2^2 h - \pi r_1^2 h \\
        &= \pi(r_2^2-r_1^2)h \\
        &= \pi(r_2 + r_1)(r_2-r_1)h \\
        &= 2\pi\frac{r_2+r_1}{2}h(r_2-r_1)
    \end{align*}
    Thus, letting \(\Delta r = r_2 - r_1\) (the thickness of the shell) and \(r = \frac{1}{2}(r_2 + r_1)\) (the average
    radius of the shell), then we find the volume to be:
    \begin{equation}
      \label{eqn1}
      V = 2\pi r h \Delta r
    \end{equation}
    Now let \(S\) be the solid obtained by rotating about the \(y\)-axis the region bounded by the curves 
    \(y = f(x)\) and \(y = g(x)\) and between \(x=a\) and \(x=b\), where \(b > a \geq 0\). We divide the interval
    \([a,b]\) into \(n\) subintervals \([x_{i-1}, x_i]\) of equal width \(\Delta x\) and let \(\bar{x}_i\) be the 
    midpoint of the \(\spscript{i}{th}\) subinterval. If the rectangle with base \([x_{i-1}, x_i]\) and height
    \(f(\bar{x}_i)\) is rotated about the \(y\)-axis, then the result is a cylindrical shell with average radius
    \(\bar{x}_i\), height \(f(\bar{x}_i)\), and thickness \(\Delta x\). Thus by \eqref{eqn1}:
    \[
      V = (2\pi\bar{x}_i)[f(\bar{x}_i)]\Delta x\text{.}
    \]
    Therefore an approximation to the volume \(V\) of \(S\) is given by the sum of these volumes of these shells. 
    We can improve this approximation arbitrarily as \(n\rightarrow\infty\), which from the definition of an
    integral, implies
    \[
      \xlimit{n}{\infty}\sum_{i=1}^n 2\pi\bar{x}_i f(\bar{x}_i)\Delta x = \int_a^b 2\pi x f(x)dx\text{.}
    \]
    \end{proof}
  
  \tbullet{6.5.1 (Average Value)}
    The "average value of \(f\)" on an interval \(a, b\) is \[ f_{\text{ave}} = \frac{1}{b-a}\int_a^b f(x)dx\text{.}\]
    
    \begin{proof}
      Divide interval \([a, b]\) into \(n\) equal subintervals, each with length \(\Delta x = \frac{b-a}{n}\). 
      Choose points \(x_1^*, x_2^*, \ldots, x_n^*\) in successive subintervals and calculate the average of the numbers
      \(f(x_1^*), f(x_2^*), \ldots, f(x_n^*)\): \[ \frac{f(x_1^*) + f(x_2^*) + \cdots + f(x_n^*)}{n}\text{.} \]
      Since \(\Delta x = \frac{b-a}{n}\), then \(n = \frac{b-a}{\Delta x}\) and thus the average becomes:
      \begin{align*}
        \frac{f(x_1^*) + f(x_2^*) + \cdots + f(x_n^*)}{\frac{b-a}{\Delta x}} &= 
          \frac{1}{b-a}\left[f(x_1^*)\Delta x + \cdots + f(x_n^*)\Delta x\right] \\
          &= \frac{1}{b-a}\sum_{i=1}^n f(x_i^*)\Delta x\text{.}
      \end{align*}
      Then by definition of definite integrals, 
      \[
        \xlimit{n}{\infty}\frac{1}{b-a}\sum_{i=1}^n f(x_i^*)\Delta x = \frac{1}{b-a}\int_a^b f(x)dx\text{.}
      \]
    \end{proof}
    
  \tbullet{6.5.2 (Mean Value Theorem for Integrals)}
    If \(f\) is continuous on \([a, b]\), then there exists a number \(c\) in \([a, b]\) such that
    \[ f(c) = f_{\text{ave}} = \frac{1}{b-a}\int_a^b f(x)dx\text{,} \] that is, \[ \int_a^b f(x)dx = 
    f(x)(b-a)\text{.} \]
    
    \begin{proof}
      Let \(f\) be continous on an interval \([a, b]\), and define \(F(x) = \int_a^x f(t)dt\). By the 
      Fundamental Theorem of Calculus, we note \(F(x)\) is continuous on \([a, b]\), differentiable on
      \(a, b\), and \(F'(x) = f(x)\). Since \(F\) is continuous on the given interval, by the Mean Value Theorem
      there exists a \(c \in (a, b)\) such that \(F(b) - F(a) = F'(c)(b - a)\). This implies:
      \begin{align*}
        & \qquad F(b) - F(a) = F'(c)(b - a) \\
        &\Rightarrow \int_a^b f(t)dt - \int_a^a f(t)dt = f(c)(b-a) \\
        &\Rightarrow \int_a^b f(t)dt = f(c)(b-a)
      \end{align*}
      completing the proof.
    \end{proof}

\end{outline}

\end{document}
