\documentclass[a4paper,8pt]{article}
\usepackage[a4paper, margin=15mm]{geometry}
\usepackage[T1]{fontenc}
\usepackage[utf8]{inputenc}
\usepackage{mathfunc}

% Header
% ===========================

\title{Integrals}
\author{Calculus Early Transcendentals by James Stewart (Chapter 5)}
\date{May 30\textsuperscript{th}, 2015}


% Document
% ===========================

\begin{document}
\maketitle
\pagenumbering{gobble}

\begin{outline}

  \dbullet{5.1.1 (Area)}
    The "area" \(A\) of the region \(S\) that lies under the graph of the continous function \(f\) is the limit of
    the sum of the areas of the approximating rectangles \(A = \xlimit{n}{\infty} = \xlimit{x}{\infty}\sum_{i=1}^n
    f(x_i)\Delta x\).

    \dbullet{5.2.1 (Definite Integral)}
    If \(f\) is a function defined for \(a \leq x \leq b\), we divide the interval \([a, b]\) into \(n\) subintervals
    of equal width \(\Delta x = \frac{b-a}{n}\). We let \(x_0 (=a), \inflatedot{x}{n} (=b)\) be the endpoints of
    these subintervals and we let \(\inflatedot{x^{*}}{n}\) be any "sample points" in these subintervals, so \(x_i^{*}\)
    lies in the \(\spscript{i}{th}\) subinterval \([x_{i-1}, x_{i}]\). Then the "definite integral of \(f\) from \(a\)
    to \(b\)" is \(\int_a^bf(x)dx = \xlimit{n}{\infty}\sum_{i=1}^n f(x_i^{*})\Delta x\) provided that this limit
    exists. If it does exist, we say that \(f\) is "integrable" on \([a, b]\).

  \tbullet{5.2.2}
    If \(f\) is continous on \([a, b]\), or if \(f\) has only a finite number of jump discontinuities, then \(f\) is
    integrable on \([a, b]\); that is, the definite integral \(\int_a^b f(x)dx\) exists.

  \tbullet{5.2.3 (Midpoint Rule)}
    \(\int_a^bf(x)dx \approx \sum_{i=1}^nf(\bar{x_i})\Delta x\) where \(\Delta x = \frac{b-a}{n}\) and
    \(\bar{x_i} = \frac{1}{2}(x_{i-1}+x_i) = \) midpont of \([x_{i-1}, x_i]\).

  \tbullet{5.3.1 (The Fundamental Theorem of Calculus)}
    \begin{enumerate}[i.]
      \item
        If \(f\) is continous on \([a, b]\), then the function \(g\) defined by \(g(x) = \int_a^x f(t)dt\) for
        \(a \leq x \leq b\) is continous on \([a, b]\), differentiable on \((a, b)\), and \(g'(x) = f(x)\).
      \item
        If \(f\) is continous on \([a, b]\), then \(\int_a^b f(x)dx = F(b)-F(a)\) where \(F\) is any antiderivative
        of \(f\); that is, a function such that \(F' = f\).
    \end{enumerate}

    \begin{proof}
      \begin{enumerate}[i.]
        \item
          If \(x\) and \(x+h\) are in \((a, b)\), then
          \begin{align*}
            g(x+h)-g(x)&=\int_a^{x+h}f(t)dt-\int_a^xf(t)dt\\
                       &=\left(\int_a^xf(t)dt + \int_x^{x+h}f(t)dt\right)-\int_a^{x}f(t)dt
                        =\int_x^{x+h}f(t)dt
          \end{align*}
          Therefore, for \(h \neq 0\),

          \begin{equation}
            \label{eqn1}
            \frac{g(x+h)-g(x)}{h} = \frac{1}{h}\int_x^{x+h}f(t)dt
          \end{equation}

          To begin, assume \(h > 0\). Since \(f\) is continous on \([x, x+h]\), the Extreme Value Theorem states
          numbers \(u, v\in [x, x+h]\) exist such that \(f(u)=m\) and \(f(v)=M\) where \(m\) and \(M\) are absolute
          minimum and maximum values of \(f\) in \([x, x+h]\). Note \(mh \leq \int_x^{x+h}f(t)dt \leq Mh
          \Rightarrow f(u)h \leq\int_x^{x+h}f(t)dt \leq f(v)h\). Since \(h > 0\), we see \(f(u) \leq
          \frac{1}{h}\int_x^{x+h}f(t)dt \leq f(v)\). By equation \eqref{eqn1} then,

          \begin{equation}
            \label{eqn2}
            f(u) \leq \frac{g(x+h)-g(x)}{h} \leq f(v)\text{.}
          \end{equation}

          A similar proof of equation \eqref{eqn2} holds for \(h < 0\). Now let \(h\rightarrow 0\) which means
          \(u\rightarrow x\) and \(v\rightarrow x\) since \(u, v\in [x, x+h]\). Therefore \(\xlimit{h}{0}f(u) =
          \xlimit{u}{x}f(u)=f(x)\) and \(\xlimit{h}{0}f(v)=\xlimit{v}{x}f(v)=f(x)\). Then by equation \eqref{eqn2}
          and the Squeeze Theorem, \(g'(x) = \xlimit{h}{0}\frac{g(x+h)-g(x)}{h} = f(x)\). If \(x=a\) or \(x=b\),
          we consider \(g'(x)\) as a one sided-limit. Regardless, because \(g\) is differentiable, then \(g\) is
          continous on \([a,b]\).
        \item
          Let \(g(x)=\int_a^xf(t)dt\), and note from (i) that \(g'(x) = f(x)\). If \(F\) is any other
          antiderivative of \(f\) on \([a, b]\), the Corollary 4.2.4 states \(F\) and \(g\) differ by a constant,
          that is,
          \begin{equation}
            \label{eqn3}
            F(x) = g(x) + C\text{ for }a < x < b\text{.}
          \end{equation}
          But both \(F\) and \(g\) are continuous on \([a, b]\) and so, by taking limits of both sides of
          \eqref{eqn3} (as \(x\rightarrow a^{+}\) and \(x\rightarrow b^{-}\)), we see that it also holds when
          \(x = a\) and \(x = b\). Also note \(g(a) = \int_a^a f(t)dt = 0\). Therefore, via \eqref{eqn3}
          with \(x=b\) and \(x=a\), we have
          \begin{align*}
            F(b)-F(a) &= [g(b) + C] - [g(a) + C] = g(b)-g(a) = g(b)\\
                      &= \int_a^bf(t)dt\text{.}
          \end{align*}
      \end{enumerate}
    \end{proof}

    \dbullet{5.4.1 (Indefinite Integral)}
    An "indefinite integral" of a function \(f\) is the antiderivative of \(f\), denoted \(\int f(x)dx=F(x)\)
    and meaning \(F'(x)=f(x)\).

  \tbullet{5.4.2}
    The integral of a rate of change is the net change, \(\int_a^bF'(x)dx = F(b)-F(a)\).

  \tbullet{5.5.1 (The Substitution Rule)}
    If \(u=g(x)\) is a differentiable function whose range is an interval \(I\) and \(f\) is continous
    on \(I\), then \(\int f(g(x))g'(x)dx = \int f(u)du\).

    \begin{proof}
      If \(F'=f\), then \(\int F'(g(x))g'(x)dx = F(g(x)) + C\), by the Chain Rule. Then setting
      \[ u = g(x), F(g(x)) + C = F(u) + C = \int F'(u)du\text{.} \] Thus
      \[ \int f(g(x))g'(x)dx = \int f(u)du\text{.} \]
    \end{proof}

  \tbullet{5.5.2 (The Substitution Rule for Definite Integrals)}
    If \(g'\) is continuous on \([a, b]\) and \(f\) is continous on the range \(u=g(x)\), then
    \(\int_a^b f(g(x))g'(x)dx = \int_{g(a)}^{g(b)} f(u)du\).

    \begin{proof}
      Let \(F\) be an antiderivative of \(f\). Then \(F(g(x))\) is an antiderivative of \(f(g(x))g'(x)\). So
      by the Fundamental Theorem of Calculus, \(\int_a^b f(g(x))g'(x)dx = F(g(b)) - F(g(a))\). We also note
      \[ \int_{g(a)}^{g(b)} f(u)du = F(u)\vert_{g(a)}^{g(b)} = F(g(b))-F(g(a))\text{.} \]
    \end{proof}

  \tbullet{5.5.3 (Integrals of Symmetric Functions)}
    Suppose \(f\) is continuous on \([-a, a]\). If \(f\) is even, then \(\int_{-a}^{a}f(x)dx = 2\int_0^a f(x)dx\).
    If \(f\) is odd, then \(\int_{-a}^a f(x)dx = 0\).

    \begin{proof}
      First note
      \[
        \int_{-a}^a f(x)dx
          = \int_{-a}^0 f(x)dx + \int_0^a f(x)dx
          = -\int_0^{-a} f(x)dx + \int_0^a f(x)dx\text{.}
      \]
      Next let \(u = -x\), implying \(du = -dx\) and when \(x = -a, u = a\). Therefore
      \[
        -\int_0^a f(x)dx = -\int_0^a f(-u)(-du) = \int_0^a f(-u)du\text{.}
      \]
      Hence, \[\int_{-a}^a f(x)dx = \int_0^a f(-u)du + \int_0^a f(x)dx\text{.}\]
      Now if \(f\) is even, \[f(-u)=f(u)\text{ and }\int_{-a}^a f(x)dx = 2\int_0^a f(x)dx\text{.}\]
      Likewise, if \(f\) is odd, \[f(-u) = -f(u)\text{ and }\int_{-a}^a f(x)dx = 0\text{.}\]
    \end{proof}

\end{outline}

\end{document}
