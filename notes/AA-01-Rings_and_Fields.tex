\documentclass[a4paper,11pt]{article}
\usepackage[a4paper, margin=20mm]{geometry}
\usepackage[T1]{fontenc}
\usepackage[utf8]{inputenc}
\usepackage{mathfunc}

% Header
% ===========================

\title{Rings and Fields}
\author{A First Course In Abstract Algebra by John B. Fraleigh (Chapter 18)}
\date{May 27\textsuperscript{th}, 2015}


% Document
% ===========================

\begin{document}
\maketitle
\pagenumbering{gobble}

\begin{outline}

  \dbullet{18.1}
    A "ring" \(\langle R, +, \cdot \rangle\) is a set \(R\) together with two binary operations
    \(+\) and \(\cdot\), which we call "addition" and "multiplication" respectively, defined on \(R\) such that the
    following axioms are satisfied:
    \begin{enumerate}[i.]
      \item \(\langle R, + \rangle\) is an ableian group.
      \item Multiplication is associative.
      \item For all \(a, b, c \in R\), the "left distributive law," \(a \cdot (b + c) = (a \cdot b) + (a \cdot c)\) and
      the "right distributive law" \((a + b) \cdot c = (a \cdot c) + (b \cdot c)\) hold.
    \end{enumerate}

  \tbullet{18.2}
    If \(R\) is a ring with additive identity \(0\), then for any \(a, b \in R\) we have:
    \begin{enumerate}[i.]
      \item \(0a = a0 = 0\)
      \item \(a(-b) = (-a)b = -(ab)\)
      \item \((-a)(-b) = ab\)
    \end{enumerate}
    
    \begin{proof}
      \begin{enumerate}[i.]
        \item 
          Note \(a0 + a0 = a(0 + 0) = a) = 0 + a0\). Then by cancellation law for \(\langle R, + \rangle\), we have 
          \(a0 = 0\). Similarly, \(0a + 0a = (0 + 0)a = 0a = 0 + 0a \Rightarrow 0a = 0\).
        \item 
          Note \(-(ab)\) is the element that when added to \(ab\) gives \(0\). Thus we show \(a(-b) + ab = 0\). But
          \(a(-b) + ab = a(-b + b) = a0 = 0\). similarly \((-a)b + ab = (-a + a)b = 0b = 0\).
        \item 
          Note \((-a)(-b) = -(a(-b)) = -(-(ab))\). This is the element that added to \(-(ab)\) is \(0\). Thus
          \((-a)(-b) = ab\).
      \end{enumerate}
    \end{proof}
      
  \dbullet{18.3}
    An "isomorphism" \(\phi: R \rightarrow R'\) from a ring \(R\) to a ring \(R'\) is a homomorphism that 
    is one-to-one and onto \(R'\). The rings \(R\) and \(R'\) are then "isomorphic."
      
  \dbullet{18.4}
    A ring in which the multiplication is commutative is a "commutative ring." A ring with a multiplicative 
    identity element is a "ring with unity;" the multiplicative identity element \(1\) is called "unity."
      
  \dbullet{18.5}
    Let \(R\) be a ring with unity \(1 \neq 0\). An element \(u\) in \(R\) is a "unit" of \(R\)
    if it has a multiplicative inverse in \(R\). If ever nonzero element of \(R\) is a unit, then \(R\) is a 
    "division ring." A "field" is a commutative division ring. A noncommutative division ring is called a "strictly
    skew field."
      
\end{outline}

\end{document}
