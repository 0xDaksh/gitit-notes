\documentclass[a4paper,8pt]{article}
\usepackage[a4paper, margin=15mm]{geometry}
\usepackage[T1]{fontenc}
\usepackage[utf8]{inputenc}
\usepackage{mathfunc}

% Header
% ===========================

\title{Fermat's and Euler's Theorems}
\author{A First Course In Abstract Algebra by John B. Fraleigh (Chapter 20)}
\date{May 28\textsuperscript{th}, 2015}


% Document
% ===========================

\begin{document}
\maketitle
\pagenumbering{gobble}

\begin{outline}

  \tbullet{20.1 (Little Theorem of Fermat)}
    If \(a \in \mathbb{Z}\) and \(p\) is a prime not dividing \(a\),
    the \(p\) divides \(a^{p-1}-1\), that is, \(a^{p-1} \equiv 1 \pmod{p}\) for \(a \neq 0 \pmod{p}\).

  \cbullet{20.2}
    If \(a \in \mathbb{Z}\), then \(a^p \equiv a \pmod{p}\) for any prime \(p\).

    \begin{proof}
      This follows from Theorem 20.1 if \(a \not\equiv 0 \pmod{p}\). If \(a \equiv 0 \pmod{p}\), then
      both sides reduce to \(0\) modulo \(p\).
    \end{proof}

  \tbullet{20.3}
    The \(G_n\) of nonzero elements of \(\mathbb{Z}_n\) that are not \(0\) divisors forms a group
    under multiplication modulo \(n\).

    \begin{proof}
      First we show \(G_n\) is a group under multiplicaiton modulo \(n\). Let \(a, b \in G_n\). If \(ab \not\in G_n\),
      then there would exist \(c \neq 0\) in \(\mathbb{Z}_n\) such that \((ab)c = 0\). Now \((ab)c = 0 \Rightarrow a(bc)
      = 0\). Sinc e\(b \in G_n\) and \(c \neq 0\), we have \(bc \neq 0\) by definition of \(G_n\). But then \(a(bc) = 0
      \Rightarrow a \not\in G_n\), contrary to assumption.

      We now show that \(G_n\) is a group. We note multiplication modulo \(n\) is associative and \(1 \in G_n\) is the
      identity. It remains to show that for \(a \in G_n\), there is a \(b \in G_n\) such that \(ab = 1\). Let \(1, a_1,
      \ldots, a_r\) be the elements of \(G_n\). The elements \(a1, aa_1, \ldots, aa_r\) are all different, for if \(aa_i
      = aa_j\), then \(a(a_i-a_j) = 0\), and \(a \in G_n\) so it isn't a divisor of \(0\). In this case then \(a_i-a_j = 0
      \Rightarrow a_i = a_j\). Therefore either \(a1 = 1\) or \(aa_i = 1\) for some \(a_i\), so \(a\) has a multipliative
      inverse.
    \end{proof}

  \dbullet{20.4}
    Let \(\phi(n)\) be defined as the number of positive integers less than or equal to \(n\) and relatively prime
    to \(n\). In other words, the number of nonzero elements of \(\mathbb{Z}_n\) that are not divisors
    of \(0\). This function \(\phi: \mathbb{Z}^{+} \rightarrow \mathbb{Z}^{+}\) is the "Euler phi-function."

  \tbullet{20.5 (Euler's Theorem)}
    If \(a\) is an integer relatively prime to \(n\), then \(a^{\phi(n)} - 1\) is divisible by \(n\),
    that is, \(a^{\phi(n)} \equiv 1 \pmod{n}\).

    \begin{proof}
      Using the fact that multiplication of these cosets by multiplication modulo \(n\) of representatives is well-defined,
      we have \(a^{\phi(n)} \equiv b^{\phi(n)} \pmod{n}\). But by Theorem 19.2 and Theorem 20.3, \(b\) can be viewed as an
      element of the multiplicative grpou \(G_n\) of order \(\phi(n)\) consisting of the \(\phi(n)\) elements of
      \(\mathbb{Z}_n\) relatively prime to \(n\). Thus \(b^{\phi(n)} \equiv 1 \pmod{n}\) and the theorem follows.
    \end{proof}

  \tbullet{20.6}
    Let \(m\) be a positive integer and let \(a \in \mathbb{Z}_m\) be relatively prime to \(m\).
    For each \(b \in \mathbb{Z}_m\), the equation \(ax = b\) has a unique solution in \(\mathbb{Z}_m\).

    \begin{proof}
      By Theorem 20.3, \(a\) is a unit in \(\mathbb{Z}_m\) and \(s = a^{-1}b\) is certainly a solution of the equation.
      Multiplying both sides of \(ax = b\) on the left by \(a^{-1}\), we see it's the only solution.
    \end{proof}

  \cbullet{20.7}
    If \(a\) and \(m\) are relatively prime integers, then for any integer \(b\), the congruence
    \(ax \equiv b \pmod{m}\) has as solutions all integers in precisely one residue class module \(m\).

  \tbullet{20.8}
    Let \(m\) be a positive integer and let \(a, b \in \mathbb{Z}_m\). Let \(d\) be the gcd of
    \(a\) and \(m\). The equation \(ax = b\) has a solution in \(\mathbb{Z}_m\) if and only if \(d\)
    divides \(b\). When \(d\) divides \(b\), the equation has exactly \(d\) solutions in \(\mathbb{Z}_m\).

    \begin{proof}
      First we show there is no solution of \(ax = b\) in \(\mathbb{Z}_m\) unless \(d\) divides \(b\). Suppose
      \(s \in \mathbb{Z}_m\) is a solution which implies \(as - b = qm\) in \(\mathbb{Z}_m\), so \(b = as - qm\).
      Since \(d\) divides both \(a\) and \(m\), we see \(d\) divides the right-hand side of the equation
      \(b = as = qm\), and hence divides \(b\). Thus a solution exists only if \(d\) divides \(b\).

      So suppose \(d\) divides \(b\). Let \(a = a_1d, b = b_1d\), and \(m = m_1d \Rightarrow as -b = qm\) in
      \(\mathbb{Z}\) is \(d(a_1s-b_1) = dqm_1\). Then \(as - b\) is a multiple of \(m\) if and only if
      \(a_1s - b_1\) is a multiple of \(m_1\).

      Thus the solutions \(s\) of \(ax = b\) in \(\mathbb{Z}_m\) are elements, read modulo \(m_1\), that yield solutions
      to \(a_1x = b_1\) in \(\mathbb{Z}_m\). Let \(s \in \mathbb{Z}_m\) be the unique solution of \(a_1x = b_1\) in
      \(\mathbb{Z}_m\) and note the numbers in \(\mathbb{Z}_m\) that reduce to \(s\) module \(m_1\) are \(s, s + m_1,
      \ldots, s + (d-1)m_1\), so \(d\) solutions.
    \end{proof}

  \cbullet{20.9}
    Let \(d\) be the gcd of positive integers \(a\) and \(m\). \(ax \equiv b \pmod{m}\) has a solution
    if and only if \(d\) divides \(b\). When this occurs, the solutions are the integers in exactly \(d\) distinct residue
    classes modulo \(m\).

\end{outline}

\end{document}
