\documentclass[a4paper,11pt]{article}
\usepackage[a4paper, margin=20mm]{geometry}
\usepackage[T1]{fontenc}
\usepackage[utf8]{inputenc}
\usepackage{mathfunc}

% Header
% ===========================

\title{Random Variables}
\author{A First Course in Probability by Sheldon Ross (Chapter 4)}
\date{June 09\textsuperscript{th}, 2015}


% Document
% ===========================

\begin{document}
\maketitle
\pagenumbering{gobble}

\begin{outline}

  \dbullet{4.1.1 (Cumulative Distribution Function)}
    For a random variable \(X\), the function \(F\) defined by \[F(x)=\bbp[X\leq x], -\infty < x < \infty\]
    is called the "cumulative distributive function," or, more simply, the "distribution function," of \(X\). 
    Thus, the distribution function specifies, for all real values \(x\), the probabiliity that the random 
    variable is less than or equal to \(x\).
    
  \dbullet{4.2.1 (Discrete Random Variables)}
    A random variable that can take on at most a countable number of possible values is said to be "discrete."
    
  \dbullet{4.2.2 (Probability Mass Function)}
    For a discrete random variable \(X\), we define the "probability mass function" \(p(a)\) of \(X\) by 
    \[p(a) = \bbp[X=a]\text{.}\] The probability mass function \(p(a)\) is positive for at most a countable 
    number of values of \(a\). That is, if \(X\) must assume one of the values \(x_1, x_2, \ldots,\) then 
    \begin{align*}
      p(x_i) &\geq 0\text{ for }i = 1, 2, \ldots \\
      p(x) &= 0\text{ for all other values of }x
    \end{align*}
    
  \dbullet{4.3.1 (Expected Value)}
    If \(X\) is a discrete random variable having a probability mass function \(p(x)\), then the "expectation," or the "expected value," of \(X\), denoted by \(E[X]\), is defined by \[E[X] = \sum_{x:p(x)>0}xp(x)\text{.}\]
    
  \dbullet{4.3.2 (Indicator Variables)}
    We say that \(I\) is an "indicator variable" for the event \(A\) if 
    \[ I = \begin{cases} 
             1 \text{ if } A \text{ occurs} \\
             0 \text{ if } A^c \text{ occurs}
       \end{cases}
    \]
    
  \pbullet{4.4.1 (Expectation of Function Compositions)}
    If \(X\) is a discrete random variable that takes on one of the values \(x_i, i\ge 1\), with respective 
    probabilities \(p(x_i)\), then, for any real-valued function \(g\), \[ E[g(X)] = \sum_i g(x_i)p(x_i)\text{.} \]
    
    \begin{proof}
      We group together all terms in \(\sum_i g(x_i)p(x_i)\) with the same value of \(g(x_i)\). Specifically, 
      suppose that \(y_j, j\geq 1\), represent the different values of \(g(x_i), i\geq 1\). Then, grouping 
      all the \(g(x_i)\) having the same value gives
      \begin{align*}
        \sum_ig(x_i)p(x_i) &= \sum_j\sum_{i: g(x_i)=y_j} g(x_i)p(x_i) \\
                           &= \sum_j\sum_{i:g(x_i)=y_j} y_jp(x_i) \\
                           &= \sum_jy_j\sum_{i:g(x_i)=y_j}p(x_i) \\
                           &= \sum_jy_j\bbp[g(X)=y_j]\\
                           &= E[g(x)]
      \end{align*}
    \end{proof}
    
  \cbullet{4.4.2 (Expectation Factors)}
    If \(a\) and \(b\) are constants, then \(E[aX + b] = aE[X] + b\).
    
    \begin{proof}
      Letting \(X\) be a random variable, and \(a, b\) be constants:
      \begin{align*}
        E[aX + b] &= \sum_{x:p(x)>0}(ax + b)p(x) \\
                  &= a\sum_{x:p(x)>0}xp(x) + \sum_{x:p(x)>0}bp(x) \\
                  &= aE[X] + b
      \end{align*}
    \end{proof}
    
  \dbullet{4.4.3 (Moments of X)}
    The expected value of a random variable \(X, E[X]\), is also referred to as the "mean" or the "first moment 
    of \(X\)." The quantity \(E[X^n], n\geq 1\), is called the "\(\spscript{n}{th}\) moment of \(X\)."
    
  \cbullet{4.4.4 (Expectation of Moments)}
    By Proposition 4.4.1, \[E[X^n] = \sum_{x:p(x)>0}x^np(x)\text{.}\]
    
  \dbullet{4.5.1 (Variance)}
    If \(X\) is a random variable with mean \(\mu\), then the variance of \(X\), denoted by \(\text{Var}(X)\) 
    is defined by \[\text{Var}(X) = E[(X-\mu)^2]\text{.} \]
    
    \begin{justification}
      Because we expect \(X\) to take on values around \(E[X]\), a reasonable means of measuring variation would be 
      \(E[|X-\mu]\), where \(\mu = E[X]\). This ends up being inconvenient to work with though, so the square of the
      difference between \(X\) and its mean is used instead.
    \end{justification}
    
  \tbullet{4.5.2 (Alternative Variance)}
    If \(X\) is a random variable with mean \(\mu\), then the variance of \(X\), denoted by \(\text{Var}(X)\) 
    is defined by \[\text{Var}(X) = E[X^2]-(E[X])^2\text{.} \]
    
    \pagebreak
    \begin{proof}
      \begin{align*}
        \text{Var}(X) &= E[(X-\mu)^2] \\
                      &= \sum_x(x-\mu)^2p(x) \\
                      &= \sum_x(x^2-2\mu x+\mu^2)p(x) \\
                      &= \sum_x x^2p(x) - 2\mu\sum_x xp(x) + \mu^2\sum_x p(x) \\
                      &= E[X^2] - 2\mu^2 + \mu^2 \\
                      &= E[X^2] - \mu^2
      \end{align*}
    \end{proof}
    
  \cbullet{4.5.3 (Variance Factors)}
    For random variable \(X\) and constants \(a\) and \(b\),
    \[ \text{Var}(aX+b) = a^2\text{Var}(X)\text{.} \]
    
    \begin{proof}
      Let \(\mu = E[X]\) and note from Corollary 4.4.2 that \(E[aX+b] = a\mu+b\).
      Therefore
      \begin{align*}
        \text{Var}(aX+b) &= E[(aX+b-a\mu-b)^2] \\
                         &= E[a^2(X-\mu)^2] \\
                         &= a^2E[(X-\mu)^2] \\
                         &= a^2\text{Var}(X)
      \end{align*}
    \end{proof}
    
  \dbullet{4.5.4 (Standard Deviation)}
    The square root of the \(\text{Var}(X)\) is called the "standard deviation of \(X\)," and is denoted by 
    \(\text{SD}(X)\). That is, \[\text{SD}(X) = \sqrt{\text{Var}(X)}\text{.}\]
    
  \dbullet{4.6.1 (Binomial Random Variables)}
    Let \(X\) be a random variable representin gthe number of successes occurring in \(n\) independent trials, 
    each of which results in a success with probability \(p\) and failure with probability \(1-p\). Then 
    \(X\) is said to be a "binomial random variable" with parameters \((n, p)\).
    
  \dbullet{4.6.2 (Bernoulli Random Variable)}
    A random variable \(X\) is said to be a "Bernoulli Random Variable" if its probability mass function is given by 
    \begin{align*}
      p(0) &= \bbp[X=0] = 1 - p \\
      p(1) &= \bbp[X=1] = p
    \end{align*}
    where \(p\), \(0 \leq p \leq 1\), is the probability that a given trial is a success. Note a Bernoulli random 
    variable can equivalently be identified as a binomial random variable with parameters \((1, p)\).
    
  \tbullet{4.6.3 (PMF of Binomial Random Variables)}
    The probability mass function of a binomial random variable having parameters \((n, p)\) is given by 
    \[ p(i) = \binom{n}{i}p^i(1-p)^{n-i}\quad i=0,1,\ldots,n\text{.} \]
    
    \begin{proof}
      The probability of any particular sequence of \(n\) independent outcomes containing \(i\) successes and 
      \(n-i\) failures is \(p^i(1-p)^{n-i}\). Since there are \(\binom{n}{i}\) different sequences of the \(n\) 
      outcomes leading to \(i\) successes and \(n-i\) failures, the theorem follows.
    \end{proof}
    
  \tbullet{4.6.4 (Expectation of Binomial Random Variable)}
    The expectation of a binomial random variable \(X\) with parameters \(n\) and \(p\) is \(E[X] = np\).
    
    \begin{proof}
      Consider the expectation of the \(\spscript{k}{th}\) moment:
      \begin{align*}
        E[X^k] &= \sum_{i=0}^ni^k\binom{n}{i}p^i(1-p)^{n-i} \\
               &= \sum_{i=1}^ni^k\binom{n}{i}p^i(1-p)^{n-i}
      \end{align*}
      Using the identity \[ i\binom{n}{i} = n\binom{n-1}{i-1}\] gives
      \begin{align*}
        E[X^k] &= np\sum_{i=1}^ni^{k-1}\binom{n-1}{i-1}p^{i-1}(1-p)^{n-i} \\
               &= np\sum_{j=0}^{n-1}(j+1)^{k-1}\binom{n-1}{j}p^j(1-p)^{n-1-j}\quad j=i-1 \\
               &= npE[(Y+1)^{k-1}]
      \end{align*}
      where \(Y\) is a binomial random variable with parameters \((n-1, p)\). Thus, 
      setting \(k=1\), we get \(E[X] = np\).
    \end{proof}
    
  \tbullet{4.6.5 (Variance of Binomial Random Variable)}
    The variance of a binomial random variable \(X\) with parameters \(n\) and \(p\) is \(\text{Var}(X) = np(1-p)\).
    
    \begin{proof}
      Set \(k = 2\) in \(E[X^k]\) solved in Theorem 4.6.4, which shows
      \begin{align*}
        E[X^2] &= npE[Y+1] \\
               &= np[(n-1)p + 1]
      \end{align*}
      Thus
      \begin{align*}
        \text{Var}(X) &= E[X^2] - (E[X])^2 \\
                      &= np[(n-1)p+1] - (np)^2 \\
                      &= np(1-p)
      \end{align*}
    \end{proof}
    
  \pbullet{4.6.6}
    If \(X\) is a binomial random variable with parameters \((n, p)\), where \(0 < p < 1\), then as \(k\) goes 
    from \(0\) to \(n\), \(\bbp[X=k]\) first increases monotonically and then decreases monotonically, reaching 
    its largest value when \(k\) is the largest integer less than or equal to \((n+1)p\).
    
    \pagebreak
    \begin{proof}
      We prove the proposition by considering \(\bbp[X=k]/\bbp[X=k-1]\) and determining for what values of 
      \(k\) it is greater or less than \(1\). Now,
      \begin{align}
        \frac{\bbp[X=k]}{\bbp[X=k-1]} &= \frac{\binom{n}{k}p^k(1-p)^{n-k}}
                                              {\binom{n}{k-1}p^{k-1}(1-p)^{n-k+1}}\nonumber \\
                                      &= \frac{(n-k+1)p}{k(1-p)}
      \end{align}
      Hence, \(\bbp[X=k] \geq \bbp[X=k-1]\) if and only if \((n-k+1)p \geq k(1-p)\), or, equivalently, if and only 
      if \(k \leq (n+1)p\).
    \end{proof}
    
  \pbullet{4.6.7 (Computation of Binomial Distribution Function)}
    Supposing \(X\) is a binomial random variable with parameters \((n,p)\),
    \[ \bbp[X=k+1] = \frac{p}{1-p}\frac{n-k}{k+1}\bbp[X=k]\text{.} \]
    
    \begin{proof}
      From proof of Theorem 4.6.6 (1), we see the above follows.
    \end{proof}
    
  \dbullet{4.7.1 (Poisson Random Variable)}
    A random variable \(X\) that takes on one of the values \(0, 1, 2, \ldots\) is said to be a 
    "Poisson" random variable with parameter \(\lambda\) if, for some \(\lambda > 0\), \[ p(i) = \bbp[X=i] 
    = e^{-\lambda}\frac{\lambda^i}{i!}\quad i=0,1,2,\ldots \]
    
  \tbullet{4.7.2 (Poisson Usage)}
      The Poisson random variable may be used as an approximation for a binomial random variable with 
      parameters \((n,p)\) when \(n\) is large and \(p\) is small enough so that \(np\) is of moderate size.
      
      \begin{proof}
        Suppose \(X\) is a binomial random variable with parameters \((n,p)\), and let \(\lambda = np\). Then 
        \begin{align*}
          \bbp[X=i] &= \binom{n}{i}p^i(1-p)^{n-i} \\
                    &= \binom{n}{i}\left(\frac{\lambda}{n}\right)^i\left(1-\frac{\lambda}{n}\right)^{n-i} \\
                    &= \frac{n(n-1)\cdots(n-i+1)}{n^i}\frac{\lambda^i}{i!}\frac{(1-\lambda/n)^n}{(1-\lambda/n)^i}\text{.}
        \end{align*}
        Now for \(n\) large and \(\lambda\) moderate
        \[
          \left(1-\frac{\lambda}{n}\right)^n \approx e^{-\lambda},\quad
          \frac{n(n-1)\cdots(n-i+1)}{n^i}\approx 1,\quad
          \left(1-\frac{\lambda}{n}\right)^i \approx 1\text{.}
        \]
        Where the first approximation derives from the definition of Euler's Number (SC.01.T3.6.4). Hence, 
        \[\bbp[X=i] \approx e^{-\lambda}\frac{\lambda^i}{i!}\text{.} \]
      \end{proof}
      
    \tbullet{4.7.3 (Expectation of Poisson Variable)}
      Let \(X\) be a Poisson variable. Then \(E[X] = \lambda\).
      
      \pagebreak
      \begin{proof}
        For Poisson variable \(X\):
        \begin{align*}
          E[X] &= \sum_{i=1}^{\infty}i\left(e^{-\lambda}\frac{\lambda^i}{i!}\right) 
               = \lambda\sum_{i=1}^{\infty}\frac{e^{-\lambda}\lambda^{i-1}}{(i-1)!} \\
               &= \lambda e^{-\lambda}\sum_{j=0}^{\infty}\frac{\lambda^j}{j!},\quad j=i-1 \\
               &= \lambda
        \end{align*}
      \end{proof}
    
    \tbullet{4.7.4 (Variance of Poisson Variable)}
      Let \(X\) be a Poisson variable. Then \(\text{Var}(X) = \lambda\).
      
      \begin{proof}
        We first compute \(E[X^2]\) as follows:
        \begin{align*}
          E[X^2] &= \sum_{i=0}^{\infty}i^2\left(e^{-\lambda}\frac{\lambda^i}{i!}\right)
                  = \lambda\sum_{i=1}^{\infty}\frac{ie^{-\lambda}\lambda^{i-1}}{(i-1)!} \\
                 &= \lambda\sum_{j=0}^{\infty}\frac{(j+1)e^{-\lambda}\lambda^j}{j!},\quad j=i-1 \\
                 &= \lambda\left[\sum_{j=0}^{\infty}\frac{je^{-\lambda}\lambda^j}{j!} + 
                    \sum_{j=0}^{\infty}\frac{e^{-\lambda}\lambda^j}{j!}\right] \\
                 &= \lambda\left[E[Y] + 1\right]
                  = \lambda(\lambda + 1)
        \end{align*}
        where \(Y\) is a poisson random variable with parameter \(\lambda\). Thus,
        \[\text{Var}(X) = E[X^2] - (E[X])^2 = \lambda(\lambda+1) - \lambda^2 = \lambda\text{.}\]
      \end{proof}
      
    \pbullet{4.7.5 (Poisson Paradigm)}
      Consider \(n\) events, with \(p_i\) equal to the probability that event \(i\) occurs, \(i=1,\ldots,n\). 
      If all the \(p_i\) are "small" and the trials are either independent or at most "weakly dependent," 
      then the number of these events that occur approximately has a Poisson distribution with mean \(\sum_{i=1}^np_i\).
      
    \pbullet{4.7.6 (Computation of Poisson Distribution Function)}
      If \(X\) is a Poisson with parameter \(\lambda\), then \[ \bbp[X=i+1] = \frac{\lambda}{i+1}\bbp[X=i]\text{.} \]
      
      \begin{proof}
        Given Poisson \(X\) with parameter \(\lambda\), then
        \[ 
          \frac{\bbp[X=i+1]}{\bbp[X=i]} 
          = \frac{e^{-\lambda}\lambda^{i+1}/(i+1)!}{e^{-\lambda}\lambda^i/i!} 
          = \frac{\lambda}{i+1}\text{.} 
        \]
      \end{proof}
  
\end{outline}

\end{document}
