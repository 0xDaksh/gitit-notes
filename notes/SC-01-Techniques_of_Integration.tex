\documentclass[a4paper,11pt]{article}
\usepackage[a4paper, margin=20mm]{geometry}
\usepackage[T1]{fontenc}
\usepackage[utf8]{inputenc}
\usepackage{mathfunc}

% Header
% ===========================

\title{Techniques of Integration}
\author{Calculus Early Transcendentals by James Stewart (Chapter 7)}
\date{June 4\textsuperscript{th}, 2015}


% Document
% ===========================

\begin{document}
\maketitle
\pagenumbering{gobble}

\begin{outline}

  \tbullet{7.1.1 (Integration by Parts)}
    For differentiable functions \(f\) and \(g\), \[ \int f(x)g'(x)dx = f(x)g(x) - \int g(x)f'(x)dx \]
    
    \begin{proof}
      The Product Rule states that if \(f\) and \(g\) are differentiable functions, then
      \[
        \leibniz{x}\left[f(x)g(x)\right] = f(x)g'(x) + g(x)f'(x)\text{.}
      \]
      Therefore
      \begin{gather*}
        \int\left[f(x)g'(x)+g(x)f'(x)\right]dx = f(x)g(x) \\
        \int f(x)g'(x)dx + \int g(x)f'(x)dx = f(x)g(x)
      \end{gather*}
      which, upon rearranging, proves the theorem.
    \end{proof}
    
  \tbullet{7.1.6 (Definite Integration by Parts)}
    Assuming \(f'\) and \(g'\) are continuous, we find
    \[
      \int_a^b f(x)g'(x)dx = f(x)g(x)\Big{\vert}_a^b - \int_a^b g(x)f'(x)dx 
    \]
    
    \begin{proof}
      By the Fundamental Theorem of Calculus, 
      \[
        \int_a^b\left[f(x)g'(x)+g(x)f'(x)\right]dx = f(a)g(a) - f(b)g(b)\text{.}
      \]
      Therefore, following a similar proof as Theorem 7.1.1,
      \begin{gather*}
        \int_a^b\left[f(x)g'(x)+g(x)f'(x)\right]dx = f(x)g(x)\Big\vert_a^b \\
        \int_a^b f(x)g'(x)dx + \int_a^b g(x)f'(x)dx = f(x)g(x)\Big\vert_a^b
      \end{gather*}
      Again, upon rearranging, we see the Theorem holds.
    \end{proof}
    
  \tbullet{7.2.1 (Strategy for Evaluating \(\int\sin^m(x)\cos^n(x)dx\))}
    \begin{enumerate}[i.]
      \item
        If the power of cosine is odd, save one cosine factor and use \(\cos^2x=1-sin^2x\) to express the remaining
        factors in terms of sine.
      \item
        If the power of sine is odd, save on sine factor and use \(\sin^2=1-\cos^2x\) to express the remaining
        factors in terms of cosine.
      \item
        If the powers of both sine and cosine are even, use the half-angle identities 
        \[\sin^2x = \frac{1}{2}[1-\cos{(2x)}] \text{ and } \cos^2x = \frac{1}{2}[1+\cos{(2x)}]\text{.}\]
    \end{enumerate}
    
  \tbullet{7.2.2 (Strategy for Evaluating \(\int\tan^m(x)\sec^n(x)dx\))}
    \begin{enumerate}[i.]
      \item
        If the power of secant is even, save a factor of \(\sec^2x\) and use \(\sec^2x=1+\tan^2x\) to express the
        remaining factors in terms of \(\tan{x}\).
      \item
        If the power of tangent is odd, save a factor of \(\sec{x}\tan{x}\) and use \(\tan^2x=\sec^2x-1\) to express
        the remaining factors in terms of \(\sec{x}\).
    \end{enumerate}

  \tbullet{7.2.3}
    \(\int\tan{(x)}dx = \ln{|\sec{x}|} + C\)
    
    \begin{proof}
      Note the following:
      \begin{align*}
        \int\tan{(x)}dx &= \int\frac{\sin{x}}{\cos{x}}dx \\
                        &= -\int\frac{1}{u}du \text{ for } u=\cos{x}\\
                        &= -\ln{|u|} + C = -\ln{|\cos{x}|} + C \\
                        &= \ln{(|\cos{x}|^{-1})} + C \\
                        &= \ln{|\sec{x}|} + C
      \end{align*}
    \end{proof}
    
  \tbullet{7.2.4}
    \(\int\sec{(x)}dx = \ln{|\sec{x}+\tan{x}|} + C\)
    
    \begin{proof}
      Note the following:
      \begin{align*}
        \int\sec{(x)}dx &= \int\sec{x}\frac{\sec{x}+\tan{x}}{\sec{x}+\tan{x}}dx \\
                        &= \int\frac{\sec^2x+\sec{x}\tan{x}}{\sec{x}+\tan{x}}dx
      \end{align*}
      If we substitute \(u=\sec{x}+\tan{x}\), then \(du=(\sec{x}\tan{x}+\sec^2{x})dx\), so the integral
      becomes \(\int(1/u)du = \ln{|u|} + C\). Thus we have: \[ \int\sec{(x)}dx = \ln{|\sec{x}+\tan{x}|} + 
      C\text{.} \]
    \end{proof}
    
  \tbullet{7.3.1 (Trigonometic Substitutions)}
    \begin{enumerate}[i.]
      \item 
        Given expression \(\sqrt{a^2-x^2}\), we use substitution \(x=a\sin{\theta}, -\frac{\pi}{2}
        \leq\theta\leq\frac{\pi}{2}\).
      \item
        Given expression \(\sqrt{a^2+x^2}\), we use substitution \(x=a\tan{\theta}, -\frac{\pi}{2}
        < \theta < \frac{\pi}{2}\).
      \item 
        Given expression \(\sqrt{x^2-a^2}\), we use substitution \(x=a\sec{\theta}, 0\leq\theta<\frac{\pi}{2}\)
        or \(\pi\leq\theta<\frac{3\pi}{2}\).
    \end{enumerate}
    
  \tbullet{7.4.1 (Partial Fractions)}
    Consider a rational function \[f(x)=\frac{P(x)}{Q(x)}\] where \(P\) and \(Q\) are polynomials. If \(f\) is
    improper, that is, \(\text{deg}(P)\geq\text{def}(Q)\), then we must take the preliminary step of dividing
    \(Q\) into \(P\) until a remainder \(R(x)\) is obtained such that \(\text{deg}(R)<\text{deg}(Q)\). The division
    statement is \[f(x)=\frac{P(x)}{Q(x)}=S(x)+\frac{R(x)}{Q(x)}\] where \(S\) and \(R\) are also polynomials.
    
    \begin{proofcases}
      \item 
        The denominator \(Q(x)\) is a product of distinct linear factors.
        
        If \(Q(x) = (a_1x+b_1)(a_2x+b_2)\cdots(a_kx+b_k)\) where no factor is repeated, then \[ \frac{R(x)}{Q(x)}
        = \frac{A_1}{a_1x+b_1} + \frac{A_2}{a_2x+b_2} + \cdots + \frac{A_k}{a_kx+b_k}\text{.} \]
      \item
        \(Q(x)\) is a product of linear factors, some of which are repeated.
        
        If the first linear factor \((a_1x+b_1)\) is repeated \(r\) times, then \[ \frac{R(x)}{Q(x)} = 
        \frac{A_1}{a_1x+b_1} + \frac{A_2}{(a_1x+b_1)^2} + \cdots + \frac{A_r}{(a_1x+b_1)^r}\text{.} \]
      \item
        \(Q(x)\) contains irreducible quadratic factors, none of which is repeated.
        
        If \(Q(x)\) has the factor \(ax^2+bx+c\), where \(b^2-rac < 0\), then, in addition to the partial fractions
        in Cases (\romannumeral 1) and (\romannumeral 2), the expression for \(R(x)/Q(x)\) will have a term
        of the form \[\frac{Ax+B}{ax^2+bx+c}\] where \(A\) and \(B\) are constants to be determined.
      \item
        \(Q(x)\) contains a repeated irreducible quadratic factor.
        
        If \(Q(x)\) has the factor \((ax^2+bx+c)^r\), where \(b^2-4ac < 0\), then instead of the single partial
        fraction in Case (\romannumeral 3), the sum \[\frac{A_1x+b_1}{ax^2+bx+c}+\frac{A_2x+B_2}{(ax^2+bx+c)^2}+
        \cdots+\frac{A_rx+B_r}{(ax^2+bx+c)^r}\] occurs in the partial fraction decomposition of \(R(x)/Q(x)\).
      
      
    \end{proofcases}

\end{outline}

\end{document}
