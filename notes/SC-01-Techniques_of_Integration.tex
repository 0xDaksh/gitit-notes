\documentclass[a4paper,11pt]{article}
\usepackage[a4paper, margin=20mm]{geometry}
\usepackage[T1]{fontenc}
\usepackage[utf8]{inputenc}
\usepackage{mathfunc}
\usepackage{calc}

% Header
% ===========================

\title{Techniques of Integration}
\author{Calculus Early Transcendentals by James Stewart (Chapter 7)}
\date{June 4\textsuperscript{th}, 2015}


% Document
% ===========================

\begin{document}
\maketitle
\pagenumbering{gobble}

\begin{outline}

  \tbullet{7.1.1 (Integration by Parts)}
    For differentiable functions \(f\) and \(g\), \[ \int f(x)g'(x)dx = f(x)g(x) - \int g(x)f'(x)dx \]
    
    \begin{proof}
      The Product Rule states that if \(f\) and \(g\) are differentiable functions, then
      \[
        \leibniz{x}\left[f(x)g(x)\right] = f(x)g'(x) + g(x)f'(x)\text{.}
      \]
      Therefore
      \begin{gather*}
        \int\left[f(x)g'(x)+g(x)f'(x)\right]dx = f(x)g(x) \\
        \int f(x)g'(x)dx + \int g(x)f'(x)dx = f(x)g(x)
      \end{gather*}
      which, upon rearranging, proves the theorem.
    \end{proof}
    
  \tbullet{7.1.6 (Definite Integration by Parts)}
    Assuming \(f'\) and \(g'\) are continuous, we find
    \[
      \int_a^b f(x)g'(x)dx = f(x)g(x)\Big{\vert}_a^b - \int_a^b g(x)f'(x)dx 
    \]
    
    \begin{proof}
      By the Fundamental Theorem of Calculus, 
      \[
        \int_a^b\left[f(x)g'(x)+g(x)f'(x)\right]dx = f(a)g(a) - f(b)g(b)\text{.}
      \]
      Therefore, following a similar proof as Theorem 7.1.1,
      \begin{gather*}
        \int_a^b\left[f(x)g'(x)+g(x)f'(x)\right]dx = f(x)g(x)\Big\vert_a^b \\
        \int_a^b f(x)g'(x)dx + \int_a^b g(x)f'(x)dx = f(x)g(x)\Big\vert_a^b
      \end{gather*}
      Again, upon rearranging, we see the Theorem holds.
    \end{proof}
    
  \tbullet{7.2.1 (Strategy for Evaluating \(\int\sin^m(x)\cos^n(x)dx\))}
    \begin{enumerate}[i.]
      \item
        If the power of cosine is odd, save one cosine factor and use \(\cos^2x=1-sin^2x\) to express the remaining
        factors in terms of sine.
      \item
        If the power of sine is odd, save on sine factor and use \(\sin^2=1-\cos^2x\) to express the remaining
        factors in terms of cosine.
      \item
        If the powers of both sine and cosine are even, use the half-angle identities 
        \[\sin^2x = \frac{1}{2}[1-\cos{(2x)}] \text{ and } \cos^2x = \frac{1}{2}[1+\cos{(2x)}]\text{.}\]
    \end{enumerate}
    
  \tbullet{7.2.2 (Strategy for Evaluating \(\int\tan^m(x)\sec^n(x)dx\))}
    \begin{enumerate}[i.]
      \item
        If the power of secant is even, save a factor of \(\sec^2x\) and use \(\sec^2x=1+\tan^2x\) to express the
        remaining factors in terms of \(\tan{x}\).
      \item
        If the power of tangent is odd, save a factor of \(\sec{x}\tan{x}\) and use \(\tan^2x=\sec^2x-1\) to express
        the remaining factors in terms of \(\sec{x}\).
    \end{enumerate}

  \tbullet{7.2.3}
    \(\int\tan{(x)}dx = \ln{|\sec{x}|} + C\)
    
    \begin{proof}
      Note the following:
      \begin{align*}
        \int\tan{(x)}dx &= \int\frac{\sin{x}}{\cos{x}}dx \\
                        &= -\int\frac{1}{u}du \text{ for } u=\cos{x}\\
                        &= -\ln{|u|} + C = -\ln{|\cos{x}|} + C \\
                        &= \ln{(|\cos{x}|^{-1})} + C \\
                        &= \ln{|\sec{x}|} + C
      \end{align*}
    \end{proof}
    
  \tbullet{7.2.4}
    \(\int\sec{(x)}dx = \ln{|\sec{x}+\tan{x}|} + C\)
    
    \begin{proof}
      Note the following:
      \begin{align*}
        \int\sec{(x)}dx &= \int\sec{x}\frac{\sec{x}+\tan{x}}{\sec{x}+\tan{x}}dx \\
                        &= \int\frac{\sec^2x+\sec{x}\tan{x}}{\sec{x}+\tan{x}}dx
      \end{align*}
      If we substitute \(u=\sec{x}+\tan{x}\), then \(du=(\sec{x}\tan{x}+\sec^2{x})dx\), so the integral
      becomes \(\int(1/u)du = \ln{|u|} + C\). Thus we have: \[ \int\sec{(x)}dx = \ln{|\sec{x}+\tan{x}|} + 
      C\text{.} \]
    \end{proof}
    
  \tbullet{7.3.1 (Trigonometic Substitutions)}
    \begin{enumerate}[i.]
      \item 
        Given expression \(\sqrt{a^2-x^2}\), we use substitution \(x=a\sin{\theta}, -\frac{\pi}{2}
        \leq\theta\leq\frac{\pi}{2}\).
      \item
        Given expression \(\sqrt{a^2+x^2}\), we use substitution \(x=a\tan{\theta}, -\frac{\pi}{2}
        < \theta < \frac{\pi}{2}\).
      \item 
        Given expression \(\sqrt{x^2-a^2}\), we use substitution \(x=a\sec{\theta}, 0\leq\theta<\frac{\pi}{2}\)
        or \(\pi\leq\theta<\frac{3\pi}{2}\).
    \end{enumerate}
    
  \tbullet{7.4.1 (Partial Fractions)}
    Consider a rational function \[f(x)=\frac{P(x)}{Q(x)}\] where \(P\) and \(Q\) are polynomials. If \(f\) is
    improper, that is, \(\text{deg}(P)\geq\text{def}(Q)\), then we must take the preliminary step of dividing
    \(Q\) into \(P\) until a remainder \(R(x)\) is obtained such that \(\text{deg}(R)<\text{deg}(Q)\). The division
    statement is \[f(x)=\frac{P(x)}{Q(x)}=S(x)+\frac{R(x)}{Q(x)}\] where \(S\) and \(R\) are also polynomials.
    
    \begin{proofcases}
      \item 
        The denominator \(Q(x)\) is a product of distinct linear factors.
        
        If \(Q(x) = (a_1x+b_1)(a_2x+b_2)\cdots(a_kx+b_k)\) where no factor is repeated, then \[ \frac{R(x)}{Q(x)}
        = \frac{A_1}{a_1x+b_1} + \frac{A_2}{a_2x+b_2} + \cdots + \frac{A_k}{a_kx+b_k}\text{.} \]
      \item
        \(Q(x)\) is a product of linear factors, some of which are repeated.
        
        If the first linear factor \((a_1x+b_1)\) is repeated \(r\) times, then \[ \frac{R(x)}{Q(x)} = 
        \frac{A_1}{a_1x+b_1} + \frac{A_2}{(a_1x+b_1)^2} + \cdots + \frac{A_r}{(a_1x+b_1)^r}\text{.} \]
      \item
        \(Q(x)\) contains irreducible quadratic factors, none of which is repeated.
        
        If \(Q(x)\) has the factor \(ax^2+bx+c\), where \(b^2-rac < 0\), then, in addition to the partial fractions
        in Cases (\romannumeral 1) and (\romannumeral 2), the expression for \(R(x)/Q(x)\) will have a term
        of the form \[\frac{Ax+B}{ax^2+bx+c}\] where \(A\) and \(B\) are constants to be determined.
      \item
        \(Q(x)\) contains a repeated irreducible quadratic factor.
        
        If \(Q(x)\) has the factor \((ax^2+bx+c)^r\), where \(b^2-4ac < 0\), then instead of the single partial
        fraction in Case (\romannumeral 3), the sum \[\frac{A_1x+b_1}{ax^2+bx+c}+\frac{A_2x+B_2}{(ax^2+bx+c)^2}+
        \cdots+\frac{A_rx+B_r}{(ax^2+bx+c)^r}\] occurs in the partial fraction decomposition of \(R(x)/Q(x)\).
      
    \end{proofcases}
    
  \tbullet{7.7.1 (Midpoint Rule)}
    An approximation of an integral can be performed as such:
    \[
      \int_a^bf(x)dx \approx M_n = \Delta{x}\left[f(\bar{x}_1) + f(\bar{x}_2) + \cdots + f(\bar{x}_n)\right]
    \]
    where
    \[
      \Delta{x}=\frac{b-a}{n}\text{ and }\bar{x}_i=\frac{1}{2}(x_{i-1}+x_i)=\text{ midpoint of }[x_{i-1}, x_i]\text{.}
    \]
    
  \tbullet{7.7.2 (Trapezoidal Rule)}
    An approximation of an integral can be performed as such:
    \[
      \int_a^bf(x)dx \approx \frac{\Delta{x}}{2}\left[f(x_0)+2f(x_1)+2f(x_2)+\cdots+2f(x_{n-1})+f(x_n)\right]
    \]
    where \(\Delta{x} = (b-a)/n\) and \(x_i = a + i\Delta{x}\).
    
    \begin{justification}
      Consider the left and right endpoint approximations and take the average:
      \begin{align*}
        \int_a^bf(x)dx &\approx \frac{1}{2}\left[\sum_{i=1}^nf(x_{i-1})\Delta{x} + \sum_{i=1}^nf(x_i)\Delta{x}\right] \\
                       &= \frac{\Delta{x}}{2}\left[\sum_{i=1}^n (f(x_{i-1})+f(x_i))\right] \\
                       &= \frac{\Delta{x}}{2}\left[(f(x_0)+f(x_1)) + (f(x_1)+f(x_2)) + \cdots + (f(x_{n-1})+f(x_n))\right] \\
                       &= \frac{\Delta{x}}{2}\left[f(x_0)+2f(x_1)+2f(x_2)+\cdots+2f(x_{n-1})+f(x_n)\right]
      \end{align*}
    \end{justification}
    
  \tbullet{7.7.3 (Error Bounds)}
    Suppose \(\norm{f''(x)} \leq K\) for \(a \leq x \leq b\). If \(E_T\) and \(E_M\) are the errors in the 
    Trapezoidal Rule and Midpoint Rule respectively, then \[ \norm{E_T} \leq \frac{K(b-a)^3}{12n^2}\text{ and }
    \norm{E_M} \leq \frac{K(b-a)^3}{24n^2}\text{.} \]
    
  \tbullet{7.7.4 (Simpson's Rule)}
    An approximation of an integral can be performed as such:
    \[
      \int_a^bf(x)dx \approx S_n = \frac{\Delta{x}}{3}\left[f(x_0) + 4f(x_1) + 2f(x_2) + 4f(x_3) + \cdots +
                                   2f(x_{n-1}) + 4f(x_{n-1}) + f(x_n)\right]
    \]
    where \(n\) is even and \(\Delta{x} = (b-a)/n\).
    
    \begin{justification}
      Divide \([a,b]\) into \(n\) subintervals of equal length \(h = \Delta{x} = (b-a)/n\), where \(n\) is an even number.
      If \(y_i = f(x_i)\), set \(P_i = (x_i, y_i)\) as the point on the curve lying above \(x_i\). Then a typical 
      parabola passes through three consective points \(P_i, P_{i+1}\), and \(P_{i+2}\). First consider the case
      where \(x_0 = -h, x_1 = 0,\) and \(x_2 = h\). We know the equation of the parabola through \(P_0, P_1,\) and \(P_2\)
      is of the form \(y = Ax^2 + Bx + C\) and so the area under this parabola if:
      \begin{align*}
        \int_{-h}^h (Ax^2+Bx+C)dx &= 2\int_0^h (Ax^2 + C) dx \\
                                  &= 2\left[ A\frac{x^3}{3} + Cx\right]_0^h \\
                                  &= 2\left( A\frac{h^3}{3} + Ch\right) = \frac{h}{3}(2Ah^3 + 6C)
      \end{align*}
      Since the parabola passes through \(P_0(-h, y_0), P_1(0, y_1),\) and \(P_2(h,y_2)\), we have
      \begin{align*}
        y_0 &= A(-h)^2 + B(-h) + C = Ah^2 - Bh + C \\
        y_1 &= C \\
        y_2 &= Ah^2 + Bh + C
      \end{align*}
      and therefore \(y_0 + 4y_1 + y_2 = 2Ah^2 + 6C\).
      Thus we can rewrite the area under the parabola as \[ \frac{h}{3}(y_0 + 4y_1 + y_2) \] and, noting that the area
      does not change if we shift it horizontally, this is the same value from \(x=x_0\) to \(x = x_2\).
      Computing the areas under all parabolas in this manner yields:
      \begin{align*}
        \int_a^bf(x)dx &\approx \frac{h}{3}(y_0+4y_1+y_2) + \frac{h}{3}(y_2+4y_3+y_4) + \cdots + 
                                \frac{h}{3}(y_{n-2}+4y_{n-1}+y_n) \\
                       &= \frac{h}{3}(y_0 + 4y_1 + 2y_2 + 4y_3 + \cdots + 2y_{n-2} + 4y_{n-1} + y_n)\text{.}
      \end{align*}
    \end{justification}
    
  \tbullet{7.7.5 (Error Bound for Simpson's Rule)}
    Suppose that \(\norm{f^{(4)}(x)} \leq K\) for \(a \leq x \leq b\). If \(E_S\) is the error involved in using
    Simpson's Rule, then \[ \norm{E_S} \leq \frac{K(b-a)^5}{180n^4}\text{.} \]
    
  \pagebreak
  \dbullet{7.8.1 (Improper Integrals of Type 1)}
    \begin{enumerate}[i.]
      \item 
        If \(\int_a^tf(x)dx\) exists for every number \(t\geq a\), then \[\int_a^{\infty}f(x)dx = 
        \xlimit{t}{\infty}\int_a^t f(x)dx\] provided this limit exists (as a finite number).
      \item 
        If \(\int_t^bf(x)dx\) exists for every number \(t \leq b\), then \[\int_{-\infty}^bf(x)dx = 
        \xlimit{t}{-\infty}f(x)dx\] provided this limit exists (as a finite number).
        
        \fbox{\begin{minipage}{\textwidth-5em}
          The proper intergrals \(\int_a^{\infty}f(x)dx\) and \(\int_{-\infty}^{b}f(x)dx\) are called "convergent"
          if the corresponding limit exists and "divergent" if the limit does not exist.
        \end{minipage}}
      \item
        If both \(\int_a^tf(x)dx\) and \(\int_t^af(x)dx\) are convergent, then we define \[\int_{-\infty}^{\infty}f(x)dx=
        \int_a^tf(x)dx + \int_t^af(x)dx\text{,}\] where \(a\) is any real number.
    \end{enumerate}
    
  \tbullet{7.8.2 (Convergence of Inverted Powers)}
    \(\int_1^{\infty}\frac{dx}{x^p}\) is convergent if \(p > 1\) and divergent if \(p \leq 1\).
    
    \begin{proof}
      For cases (1) and (2) (i.e. \(p \neq 1\)), note the following:
      \begin{align*}
        \int_1^{\infty}\frac{1}{x^p} dx &= \xlimit{t}{\infty}\int_1^tx^{-p} dx \\
                                        &= \xlimit{t}{\infty}\frac{x^{-p+1}}{-p+1}\Big\vert_{x=1}^{x=t} \\
                                        &= \xlimit{t}{\infty}\frac{1}{1-p}\left[\frac{1}{t^{p-1}}-1\right]
      \end{align*}
      \begin{proofcases}
        \item \(p > 1\)\\
          Then \(p-1 > 0\), so as \(t\rightarrow\infty, t^{p-1}\rightarrow\infty\), and \(1/t^{p-1}\rightarrow 0\).
          Therefore \[ \int_1^{\infty}\frac{1}{x^p}dx = \frac{1}{p-1}\text{ if }p > 1 \] and so the integral converges.
        \item \(p < 1\)\\
          Then \(p-1<0\) and so \[\frac{1}{t^{p-1}} = t^{1-p}\rightarrow\infty\text{ as }t\rightarrow\infty\] and
          the integral diverges.
        \item \(p = 1\)
          \[
            \int_1^{\infty}\frac{1}{x}dx = \xlimit{t}{\infty}\int_1^t\frac{1}{x}dx 
                                         = \xlimit{t}{\infty}\ln{\norm{x}}\Big\vert_1^t
                                         = \xlimit{x}{\infty}(\ln{t}-\ln{1}) 
                                         = \xlimit{t}\ln{t} = \infty\text{.}
          \]
          The limit does not exist as a finite number so the improper integral is divergent.
      \end{proofcases}
    \end{proof}
    
  \pagebreak
  \dbullet{7.8.3 (Improper Integrals of Type 2)}
    \begin{enumerate}[i.]
      \item 
        If \(f\) is continous on \([a, b)\) and is discontinous at \(b\), then
        \[\int_a^{b}f(x)dx = \xlimit[-]{t}{b}\int_a^t f(x)dx\] if this limit exists (as a finite number).
      \item
        If \(f\) is continous on \((a, b]\) and is discontinuous at \(a\), then
        \[ \int_a^bf(x)dx = \xlimit[+]{t}{a}\int_t^bf(x)dx \] if this limit exists (as a finite number).
        
        \fbox{\begin{minipage}{\textwidth-5em}
          The improper integral \(\int_a^b f(x)dx\) is called "convergent" if the corresponding limit 
          exists and "divergent" if the limit does not exist.
        \end{minipage}}
      \item
        If \(f\) has a discontinuity at \(c\), where \(a < c < b\), and both \(\int_a^cf(x)dx\) and
        \(\int_c^bf(x)dx\) are convergent, then we define \[\int_a^bf(x)dx = \int_a^cf(x)dx + \int_c^bf(x)dx\text{.}\]
    \end{enumerate}
    
  \tbullet{7.8.4 (Comparison Theorem)}
    Suppose that \(f\) and \(g\) are continuous functions with \(f(x) \geq g(x) \geq 0\) for \(x \geq a\).
    \begin{enumerate}[i.]
      \item If \(\int_a^{\infty}f(x)dx\) is convergent, then \(\int_a^{\infty}g(x)dx\) is convergent.
      \item If \(\int_a^{\infty}g(x)dx\) is divergent, then \(\int_a^{\infty}f(x)dx\) is divergent.
    \end{enumerate}
    A similar argument holds for Type 2 improper integrals.

\end{outline}

\end{document}
