\documentclass[a4paper,11pt]{article}
\usepackage[a4paper, margin=20mm]{geometry}
\usepackage[T1]{fontenc}
\usepackage[utf8]{inputenc}
\usepackage{mathfunc}

% Header
% ===========================

\title{Appendix: Trigonometry}
\author{Calculus Early Transcendentals by James Stewart (Appendix D)}
\date{June 4\textsuperscript{th}, 2015}


% Document
% ===========================

\begin{document}
\maketitle
\pagenumbering{gobble}

\begin{outline}

  \tbullet{D.1 (Trigonometric Identities)}
    \begin{enumerate}[i.]
      \item \(\sin^2\theta + \cos^2\theta = 1\)
      \item \(\sec^2\theta - \tan^2\theta = 1\)
      \item \(\csc^2\theta - \cot^2\theta = 1\)
      \item \(\sin{(-\theta)} = -\sin{\theta}\)
      \item \(\cos{(-\theta)} = \cos{\theta}\)
      \item \(\sin{(\theta + 2\pi)} = \sin{\theta}\)
      \item \(\cos{(\theta + 2\pi)} = \cos{\theta}\)
    \end{enumerate}
    
  \tbullet{D.2 (Law of Cosines)}
    If a triangle has sides with lengths \(a, b,\) and \(c,\) and \(\theta\) is the angle between the sides
    with lengths \(a\) and \(b\), then \(c^2 = a^2 + b^2 - 2ab\cos{\theta}\).
    
    \begin{proof}
      Let sides \(A, B,\) and \(C\) be sides with lengths \(a, b,\) and \(c\) respectively. Establish a coordinate 
      system where \(A\) and \(B\) meet at the origin, and \(\theta\) is the angle between \(A\) and \(B\). Also, 
      suppose \(P(x, y)\) is the point where \(C\) meets \(B\). Then we have a triangle with points \((0, 0), (a, 0),\)
      and \((x, y)\).
      
      \begin{proofcases}
        \item
          Suppose \(x = a\).\\
          Then the angle formed between \(A\) and \(B\) is \(\frac{\pi}{2}\) and so \(-2ab\cos{\theta} = 0\),
          leaving the Pythagorean Theorem which we note is true.
        \item
          Suppose \(0 < x < a\).\\
          Then the line spanning \(P\) to the \(x\)-axis forms two right triangles, and the triangle has height
          \(b\sin{\theta}\), and a leg of length \(b\cos{\theta}\). Thus, by the Pythagorean Theorem,
          \begin{align*}
            c^2 &= (a-b\cos{\theta})^2 + (b\sin{\theta}^2) \\
                &= a^2 - 2ab\cos^2{\theta} + b^2\cos^2{\theta} + b^2\sin^2{\theta}\\
                &= a^2 - 2ab\cos^2{\theta} + b^2(\cos^2{\theta} + \sin^2{\theta}) \\
                &= a^2 + b^2 - 2ab\cos^2{\theta}
          \end{align*}
      \end{proofcases}
    \end{proof}
    
  \tbullet{D.2 (Addition Formulas)}
    \begin{enumerate}[i.]
      \item \(\sin{(x+y)} = \sin{x}\cos{y} + \cos{x}\sin{y}\)
      \item \(\cos{(x+y)} = \cos{x}\cos{y} - \sin{x}\sin{y}\)
      \item \(\tan{(x+y)} = \frac{\tan{x}+\tan{y}}{1-\tan{x}\tan{y}}\)
    \end{enumerate}
   
  \pagebreak 
  \tbullet{D.3 (Subtraction Formulas)}
    \begin{enumerate}[i.]
      \item \(\sin{(x-y)} = \sin{x}\cos{y} - \cos{x}\sin{y}\)
      \item \(\cos{(x-y)} = \cos{x}\cos{y} + \sin{x}\sin{y}\)
      \item \(\tan{(x-y)} = \frac{\tan{x}-\tan{y}}{1+\tan{x}\tan{y}}\)
    \end{enumerate}
    
    \begin{proof}
      Replace \(y\) for \(-y\) in the Addition Formulas.
    \end{proof}
    
  \tbullet{D.4 (Double-Angle Formulas)}
    \begin{enumerate}[i.]
      \item \(\sin{2x} = 2\sin{x}\cos{x}\)
      \item \(\cos{2x} = \cos^2 x - \sin^2 x\)
    \end{enumerate}
    
    \begin{proof}
      Replace \(y\) for \(x\) in the Addition Formulas.
    \end{proof}
    
  \tbullet{D.5 (Alternate Double-Angle Formulas)}
    \begin{enumerate}[i.]
      \item \(\cos{2x} = 2\cos^2 x - 1\)
      \item \(\cos{2x} = 1 - 2\sin^2 x\)
    \end{enumerate}
    
    \begin{proof}
      Use identity \(\sin^2 x + \cos^2 x = 1\) in the Double-Angle Formulas.
    \end{proof}
    
  \tbullet{D.6 (Half-Angle Formulas)}
    \begin{enumerate}[i.]
      \item \(\cos^2 x = \frac{1 + \cos{2x}}{2}\)
      \item \(\sin^2 x = \frac{1 - \cos{2x}}{2}\)
    \end{enumerate}
    
    \begin{proof}
      Solve for \(\cos^2 x\) and \(\sin^2 x\) in the Alternate Double-Angle Formulas.
    \end{proof}
    
  \tbullet{D.6 (Product Formulas)}
    \begin{enumerate}[i.]
      \item \(\sin{x}\cos{y} = \frac{1}{2}\left[\sin{(x+y)} + \sin{(x-y)}\right]\)
      \item \(\cos{x}\cos{y} = \frac{1}{2}\left[\cos{(x+y)} + \cos{(x-y)}\right]\)
      \item \(\sin{x}\sin{y} = \frac{1}{2}\left[\cos{(x-y)} - \cos{(x+y)}\right]\)
    \end{enumerate}

    \begin{proof}
      Solve for respective products with respect to Addition and Subtraction formulas.
    \end{proof}

\end{outline}

\end{document}
