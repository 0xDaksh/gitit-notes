\documentclass[a4paper,11pt]{article}
\usepackage[a4paper, margin=20mm]{geometry}
\usepackage[T1]{fontenc}
\usepackage[utf8]{inputenc}
\usepackage{mathfunc}

% Header
% ===========================

\title{Applications of Differentiation}
\author{James Stewart, Calculus Early Transcendentals (Chapter 2)}
\date{May 30\textsuperscript{th}, 2015}


% Document
% ===========================

\begin{document}
\maketitle
\pagenumbering{gobble}

\begin{outline}

  \dbullet{4.1.1}
    A function \(f\) has an "absolute maximum" at \(c\) if \(f(c) \geq f(x)\) for all \(x \in D\), where \(D\) is the
    domain of \(f\). The number \(f(c)\) is called the "maximum value" of \(f\) on \(D\). A similar definition exists
    for absolute minimums and minimal values. The maximum and minimum values of \(f\) are the "extreme values."

\end{outline}

\end{document}
