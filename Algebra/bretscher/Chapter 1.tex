\documentclass[a4paper,11pt]{article}
\usepackage[a4paper, margin=20mm]{geometry}
\usepackage[T1]{fontenc}
\usepackage[utf8]{inputenc}
\usepackage{mathfunc}

% Header
% ===========================

\title{Linear Equations}
\author{Otto Bretscher, Linear Algebra With Applications (Chapter 1)}
\date{May 27\textsuperscript{th}, 2015}


% Document
% ===========================

\begin{document}
\maketitle
\pagenumbering{gobble}

\begin{outline}

  \tbullet{1.3.1}
    A system of equations is said to be "consistent" if there is at least one solution; it is "inconsistent" 
    if there are no solutions. A linear system is inconsistent if and only if the reduced row-echelon form
    of its augmented matrix contains the row \([0\; 0 \ldots 0\; \vert\; 1]\), representing the equation \(0 = 1\). 
    If a linear system is consistent, then it has either:
    \begin{enumerate}[i.]
      \item infinitely many solutions (at least one free variable) or
      \item exactly one solution (if all variables are leading
    \end{enumerate}
    
  \dbullet{1.3.2}
    The "rank" of a matrix \(A\) is the number of leading ones in rref(\(A\)).
    
  \tbullet{1.3.3}
    A linear system with fewer equations than unknowns has either no solutions or infinitely many solutions.

  \tbullet{1.3.4}
    A linear system of \(n\) equations in \(n\) variables has a unique solution if and only if the rank of its 
    coefficient matrix \(A\) is \(n\). In this case, rref(\(A\)) is an \(n \times n\) matrix with ones along the 
    diagonal and zeros everywhere else.
      
  \dbullet{1.3.5}
    The "sum of two matrices" of the same size is defined entry by entry:
    \[
      \begin{bmatrix} a_{11} & \ldots & a_{1m} \\ \vdots & & \vdots \\ a_{n1} & \ldots & a_{nm} \end{bmatrix} +
      \begin{bmatrix} b_{11} & \ldots & b_{1m} \\ \vdots & & \vdots \\ b_{n1} & \ldots & b_{nm} \end{bmatrix} =
      \begin{bmatrix}
        a_{11} + b_{11} & \ldots & a_{1m} + b_{1m} \\
        \vdots          &        & \vdots          \\
        a_{n1} + b_{n1} & \ldots & a_{nm} + b_{nm} 
      \end{bmatrix}
    \]
      
  \dbullet{1.3.6}
    Let \(\vec{v}, \vec{w}\) be vectors with components \(\inflatedot{v}{n}\) and \(\inflatedot{w}{n}\)
    respectively. The "dot product" of \(\vec{v}\) and \(\vec{w}\), denoted \(\vec{v} \cdot \vec{w}\), is the 
    scalar \(v_1w_1 + v_2w_2 + \ldots + v_nw_n\).
    
  \dbullet{1.3.7}
    If \(A\) is an \(n \times m\) matrix with row vectors \(\inflatedot{w}{n}\), and \(\vec{x}\) is a vector 
    in \(\bbr^m\), then the \(\spscript{i}{th}\) component of \(A\vec{x}\) is the dot product of the 
    \(\spscript{i}{th}\) row of \(A\) with \(\vec{x}\).
    
  \tbullet{1.3.8}
    Let \(A\) be an \(n \times m\) matrix with column vectors \(\inflatedot{\vec{v}}{m}\). Let \(\vec{x}\) be a 
    vector in \(\bbr^m\) with components \(x_1, \ldots, x_m\). Then \(A\vec{x} = x_1\vec{v}_1 + \ldots + 
    x_m\vec{v}_m\).
    
    \begin{proof}
      Denote the rows of \(A\) by \(\inflatedot{\vec{w}}{n}\) and the entries by \(a_{ij}\).
      We need to show the \(i^{\text{th}}\) component of \(x_1\vec{v}_1 + \ldots + x_m\vec{v}_m\), 
      for \(i = 1, \ldots, n\). Now:
      \begin{align*}
        (\spscript{i}{th}\text{ component of }A\vec{x}) &= \vec{w}_i \cdot \vec{x}\\
                                                        &= a_{i1}x_1 + \ldots + a_{im}x_m\\
                                                        &= x_1(\spscript{i}{th}\text{ component of }\vec{v}_1) + \ldots +
                                                           x_m(\spscript{i}{th}\text{ component of }\vec{v}_m)
      \end{align*}
      This in turn is just the \(\spscript{i}{th}\) component of \(x_1\vec{v}_1 + \ldots + x_m\vec{v}_m\) by the
      fact that vector addition and scalar multiplication are defined component by component.
    \end{proof}

  \dbullet{1.3.9}
    A vector \(\vec{b}\) in \(\bbr^m\) is called a linear combination of the vectors
    \(\inflatedot{\vec{v}}{m}\) in \(\bbr^n\) if there exist scalars \(x_1, \ldots, x_m\) 
    such that \(\vec{b} = x_1\vec{v}_1 + \ldots + x_m\vec{v}_m\).
      
  \tbullet{1.3.10}
    If \(A\) is an \(n \times m\) matrix, \(\vec{x}\) and \(\vec{y}\) are vectors in \(\bbr^m\), and \(k\) is a 
    scalar, then:
    \begin{enumerate}[i.]
      \item \(A(\vec{x} + \vec{y}) = A\vec{x} + A\vec{y}\)
      \item \(A(k\vec{x}) = k(A\vec{x})\)
    \end{enumerate}
      
  \tbullet{1.3.1}
    We can write the linear system with augmented matrix \([A\;\vert\;\vec{b}]\) in matrix
    form as \(A\vec{x} = \vec{b}\).

\end{outline}

\end{document}
