\documentclass[a4paper,11pt]{article}
\usepackage[a4paper, margin=20mm]{geometry}
\usepackage[T1]{fontenc}
\usepackage[utf8]{inputenc}
\usepackage{mathfunc}

% Header
% ===========================

\title{Group Action on a Set}
\author{John B. Fraleigh, A First Course In Abstract Algebra (Chapter 16)}
\date{May 27\textsuperscript{th}, 2015}


% Document
% ===========================

\begin{document}
\maketitle
\pagenumbering{gobble}
\setlength\parindent{24pt}

\begin{outline}

  \dbullet{16.1}
    Let \(X\) be a set and \(G\) a group. An "action of \(G\) on \(X\)" is a map 
    \(*: G \times X \rightarrow X\) such that:
    \begin{enumerate}[i.]
      \item \(ex = x\) for all \(x \in X\),
      \item \((g_{1}g_{2})(x) = g_{1}(g_{2}x)\) for all \(x \in X\) and all \(g_{1}, g_{2} \in G\).
    \end{enumerate}
    Under these conditions, \(X\) is a "\(G\)-set."

  \tbullet{16.2}
    Let \(X\) be a \(G\)-set. For each \(g \in G\), the function \(\sigma_{g}: X \rightarrow X\)
    defined by \(\sigma_{g}(x) = gx\) for \(x \in X\) is a permutation of \(X\). Also, the map 
    \(\phi: G \rightarrow S_{X}\) defined by \(\phi(g) = \sigma_{g}\) is a homomorphism with the 
    property that \(\phi(g)(x) = gx\).
    
    \begin{proof}
      To show \(\sigma_{g}\) is a permutation of \(X\), we show it's a one-to-one map of \(X\) onto itself. Suppose
      \(\sigma_{g}(x_{1}) = \sigma_{g}(x_{i})\) for \(x_{1}, x_{2} \in X\). then \(gx_{1} = gx_{2} \Rightarrow 
      g^{-1}(gx_{1}) = (g^{-1}g)x_{1} = (g^{-1}g)x_{2} \Rightarrow x_{1} = x_{2}\). We also see \(\sigma_{g}(g^{-1}x) =
      (gg^{-1})x = x\), so \(\sigma_{g}\) maps \(X\) onto \(X\). Thus \(\sigma_{g}\) is a permuation.
      
      To show \(\phi: G \rightarrow S_{X}\) is a homomorphism defined by \(\phi(g) = \sigma_{g}\), we must show 
      \(\phi(g_{1}g_{2}) = \phi(g_{1})\phi(g_2)\) for all \(g_1, g_2 \in G\). We show equality by showing they carry an
      \(x \in X\) into the same element. Note \(\phi(g_1g_2)(x) = \sigma_{g_1g_2}(x) = (g_1g_2)x = g_1(g_2x) 
      = g_1(\sigma_{g_2}(x)) = \sigma_{g_1}(\sigma_{g_2}(x))\). Thus \(\phi(g_1g_2)(x) = 
      (\sigma_{g_1} \circ \sigma_{g_2})(x) = (\sigma_{g_1}\sigma_{g_2})(x) = (\phi(g_1)\phi(g_2))(x)\). 
      
      Thus \(\phi\) is a homomorphism.
    \end{proof}

  \dbullet{16.3}
    Let \(X\) be a \(G\)-set. We say \(G\) "acts faithfully" on \(X\) if \(\{g \in G : gx = x \forall x
    \in X\} = \{e\}\). We say \(G\) is "transitive" on \(X\) if \(\forall x_1, x_2 \in X, \exists g \in G\) 
    such that \(gx_1 = x_2\).
    
  \dbullet{16.4}
    Let \(X\) be a \(G\)-set of a group \(G\). The group \(G_x \coloneqq \{g \in G : gx = x\}\) is
    the "isotropy group" of \(X\), otherwise known as the "stabilizer," where \(x \in G\). Similarly, 
    we define \(X_g \coloneqq \{x \in X : gx = x\}\).
      
  \tbullet{16.5}
    Let \(X\) be a \(G\)-set. Then \(G_x\) is a subgroup of \(G\) for each \(x \in X\).
    
    \begin{proof}
      Let \(x \in X\) and \(g_1, g_2 \in G_x\). Then \(g_1x = g_2x = x \Rightarrow (g_1g_2)x = x\) so \(g_1g_2 \in G_x\).
      Since \(ex = x, e \in G_x\) and if \(g \in G\), then \((g^{-1}g)(x) = g^{-1}(gx) = g^{-1}x = x \Rightarrow g^{-1}
      \in G_x\).
    \end{proof}
  
  \tbullet{16.6}
    Let \(X\) be a \(G\)-set. For \(x_1, x_2 \in X\), let \(x_1\) \textasciitilde{} \(x_2\) if and
    only if there exists \(g \in G\) such that \(gx_1 = x_2\). Then \textasciitilde{} is an equivalence 
    relation on \(X\).
    
    \begin{proof}
      For each \(x \in X\), we have \(ex = x\), so \(x\) \textasciitilde{} \(x\) and \(x\) is reflexive. Suppose
      \(x_1\) \textasciitilde{} \(x_2\), so \(gx_1 = x_2\) for some \(g \in G\). Then \(g^{-1}x_1 = x_2\) so \(x_2\)
      \textasciitilde{} \(x_1\) and \textasciitilde{} is symmetric. Finally, if \(x_1\) \textasciitilde{} \(x_2\) and
      \(x_2\) \textasciitilde{} \(x_3\), then \(g_1x_1 = x_2\) and \(g_2x_2 = x_3\) for some \(g_1, g_2 \in G\)
      implying \((g_2g_1)x_1 = g_2(g_1x_1) = g_2x_2 = x_3\). So \(x_1\) \textasciitilde{} \(x_3\) and \textasciitilde{}
      is transitive.
    \end{proof}
      
  \dbullet{16.7}
    Let \(X\) be a \(G\)-set. Each cell in the partition of the equivalence relation described in Theorem 16.6 
    is an "orbit in \(X\) under \(G\)." If \(x \in X\), the cell containing \(x\) is the "orbit of \(x\)." We let 
    this cell be \(Gx\).
      
  \tbullet{16.8}
    Let \(X\) be a \(G\)-set and let \(x \in X\). Then \(|Gx| = [G:G_x]\). If \(|G|\) is finite, then \(|Gx|\) 
    is a divisor of \(|G|\).
    
    \begin{proof}
      We define a one-to-one map \(\psi\) from \(Gx\) onto the collection of left cosets of \(G_x\) in \(G\). Let \(x_1
      \in Gx\). Then \(\exists g_1 \in G\) such that \(g_1x = x_1\). We define \(\psi(x_1)\) to be the left coset \(g_1G_x\)
      of \(G_x\). Suppose also \(g_1'x = x_1\) so \(g_1x = g_1'x \Rightarrow g_1^{-1}(g_1x) = g_1^{-1}(g_1'x) \Rightarrow
      x = (g_1^{-1}g_1')x\). Thus \(g_1^{-1}g_1' \in G_x \Rightarrow g_1' \in g_1G_x\) and \(g_1G_x = g_1'G_x\) so \(\psi\)
      is well defined.
      
      To show \(\psi\) is one-to-one, suppose \(x_1, x_2 \in Gx\), and \(\psi(x_1) = \psi(x_2)\). Then
      \(\exists g_1, g_2 \in G\) such that \(x_1 = g_1x, x_2 = g_2x\), and \(g_2 \in g_1G_x\). Then \(g_2 = g_1g\) for
      some \(g \in G_x\), so \(x_2 = g_2x = (g_1g)x = g_1x = x_1\). So \(\psi\) is one-to-one.
      
      Last we show each left coset of \(G_x\) in \(G\) is of form \(\psi(x_1)\) for some \(x_1 \in Gx\). 
      Let \(g_1G_x\) be a left coset. Then if \(g_1x = x_1\), we have \(g_1G_x = \psi(x_1)\). Thus \(\psi\) 
      maps \(Gx\) one-to-one onto the collection of right cosets so \(|Gx| = [G:G_x]\). If \(|G|\) is finite, 
      then \(|G| = |G_x|[G:G_x]\) shows \(|Gx| = [G:G_x]\) is a divisor of \(|G|\).
    \end{proof}

\end{outline}

\end{document}
