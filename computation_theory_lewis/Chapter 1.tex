\documentclass[a4paper,11pt]{article}
\usepackage[a4paper, margin=20mm]{geometry}
\usepackage[T1]{fontenc}
\usepackage[utf8]{inputenc}
\usepackage{mathfunc}

% Header
% ===========================

\title{Sets, Relations, and Languages}
\author{Harry R. Lewis \& Christos H. Papadimitrious, Elements of the Theory of Computation (Chapter 1)}
\date{May 30\textsuperscript{th}, 2015}


% Document
% ===========================

\begin{document}
\maketitle
\pagenumbering{gobble}

\begin{outline}

  \dbullet{1.3.1}
    A relation \(R\) is "antisymmetric" if whenever \((a,b)\in R\) and \(a\) and \(b\) are distinct, then
    \((b, a) \not\in R\).
    
  \dbullet{1.3.2}
    A relation that is symmetric, reflexive, and transitive is an "equivalence relation." Clusters of an
    equivalence relation are called "equivalence classes," denoted \([a]\), where \(a\) is an element of the
    given equivalence class.
    
  \tbullet{1.3.3}
    Let \(R\) be an equivalence relation on a nonempty set \(A\). Then the equivalence classes of \(R\) constitute
    a partition of \(A\).
    
    \begin{proof}
      Let \(\Pi = \{[a] : a \in A\}\) which we'll show are nonempty, disjoint, and exhaustive. All equivalence classes
      are nonempty since \(a \in [a]\) for all \(a \in A\) by reflexivity. Consider two distinct equivalence classes 
      \([a]\) and \([b]\) and suppose \([a] \cap [b] \neq \emptyset\). Then there is an element \(c\) such that 
      \(c \in [a]\) and \(c \in [b]\). Hence, by transitivity, \((a, c)\) and \((c, b) \in R\) which implies 
      \((a, b) \in R\), and since \(R\) is symmetric, \((b, a) \in R\).
      
      Now take any element \(d \in [a] \Rightarrow (d, a)\in R \Rightarrow (d, b) \in R\). Thus \(d \in [b]\) so 
      \([a] \subseteq [b]\) which, by a similar argument, shows \([b] \subseteq [a]\). But this contradicts
      \([a]\) nd \([b]\) being distinct. Lastly, we see for all \(a \in A, a \in [a]\) so \(\bigcup\Pi = A\),
      and is therefore exhaustive.
    \end{proof}
    
  \dbullet{1.3.4}
    A relation that is reflexive, antisymmetric, and transitive is a "partial order." If \(R \subseteq A \times A\)
    is a partial order, an element \(a \in A\) is "minimal" if \((b, a)\in R\) only if \(a=b\). A partial order 
    \(R \subseteq A\times A\) is a "total order" if for all \(a, b \in A\), either \((a, b)\in R\) or 
    \((b, a)\in R\).
    
  \tbullet{1.5.1}
    Let \(R\) be a binary relation on a finite set \(A\), and let \(a, b\in A\). If there is a path from \(a\) to 
    \(b\) in \(R\), then there is a path of length at most \(|A|\).
    
    \begin{proof}
      Suppose \((\inflatedot{a}{n})\) is the shortest path from \(a_1 = a\) to \(a_n = b\), and suppose \(n > |A|\).
      By the pigeonhole principle, there is an element that repeats on the path. If \(a_i = a_j\), then
      \((\inflatedot{a}{i}, a_{j+1}, \ldots, a_n)\) is a shorter path, which is a contradiction.
    \end{proof}
    
  \tbullet{1.5.2 (Diagonalization Principle)}
    Let \(R\) be a binary relation on a set \(A\), and let \(D\), the diagonal set for \(R\), be
    \(\{a : a \in A \text{ and } (a, a) \not\in R\}\). For each \(a \in A\), let \(R_a = \{b : b \in A \text{ and }
    (a,b)\in R\}\). Then \(D\) is distinct from each \(R_a\).
    
    \begin{proof}
      We note \(D\) contains all non-reflexive values \(a \in A\). For every given \(a \in D\), \(a \not\in R_a\)
      since \((a, a) \not\in R\). For every given \(a \not\in D\), \(a \in R_a\) since \((a, a) \in R\). Thus
      \(D\) is distinct from all \(R_a\)'s.
    \end{proof}
    
  \tbullet{1.5.3}
    The set \(2^{\bbn}\) is uncountable.
    
    \begin{proof}
      Suppose \(2^{\bbn}\) is countably infinite, that is, there is a way to enumerate all members of \(2^{\bbn}\) as
      \(\{R_0, R_1, \ldots\}\) where \(R_a = \{b : b \in \bbr \text{ and } (a, b)\in R\}\) and \(R\) is the relation
      \(\{(i,j) : j \in R_i\}\). Now consider the diagonal set \(D=\{n\in\bbn : n\not\in R_n\}\), a subset of \(\bbn\).
      Then by the Diagonalization Principle, \(D\) is distinct from each \(R_a\) and thus does not appear in the enumeration.
      Therefore \(2^{\bbn}\) is uncountable.
    \end{proof}
    
  \dbullet{1.6.1}
    Let \(R \subseteq A^2\) be a directed graph defined on a set \(A\). The "reflexive transitive closure" of \(R\)
    is the relation \(R^* = \{(a, b): a, b \in A\text{ and there is a path from }a\text{ to }b\text{ in }R\}\).
    
  \dbullet{1.6.2}
    Let \(f:\bbn\rightarrow\bbn\). The "order of \(f\)," denoted \(O(f)\), is the set of all functions 
    \(g:\bbn\rightarrow\bbn\) with the property that there are positive natural numbers \(c\) and \(d\) such that, for
    all \(n \in \bbn\), \(g(n) \leq c\cdot f(n)+d\). We then say "\(g(n)\in O(f(n))\) with constants \(c\) and \(d\)."
    If for two functions \(f, g:\bbn\rightarrow\bbn\) we have \(f \in O(g)\) and \(g \in O(f)\), then we write
    \(f\asymp g\). All functions from the set of natural numbers to itself are partitioned by \(\asymp\) into
    equivalence classes. The equivalence class of \(f\) with respect to \(\asymp\) is called the "rate growth" of \(f\).
    
  \dbullet{1.6.3}
    Let \(D\) be a set, \(n \geq 0\), and \(R \subseteq D^{n+1}\) be a \((n+1)\)-ary relation on \(D\). Then
    a subset \(B\) of \(D\) is said to be "closed under \(R\)" if \(b_{n+1} \in B\) whenever \(\inflatedot{b}{n}\in B\)
    and \((\inflatedot{b}{n}, b_{n+1})\in R\). Any property of the form "the set \(B\) is closed under relations 
    \(\inflatedot{R}{m}\)" is called a "closure property" of \(B\).
    
  \dbullet{1.6.4}
    Let \(A, B\) be sets. \(B\) is the "minimal set that contains \(A\) and has property \(P\)" if it does not properly
    include any other set \(B'\) that also contains \(A\) and has property \(P\), and \(B\) is well-defined (there exists
    only one such minimal set).
    
  \tbullet{1.6.5}
    Let \(P\) be a closure property defined by relations on a set \(D\), and let \(A\) be a subset of \(D\). Then there
    is a unique minimal set \(B\) that contains \(A\) and has property \(P\).
    
    \begin{proof}
      Consider the set of all subsets of \(D\) that are closed under \(\inflatedot{R}{m}\) and that have \(A\) as
      a subset, which we call \(S\). We must show \(S\) has a unique minimal element. We note \(S\) is nonempty since
      \(D \in S\), itself closed under each \(R_i\) and certainly containing \(A\). 
      
      Consider the set \(B\), the intersection of all sets in \(S\), \(B = \bigcap S\). \(B\) is well-defined 
      since it is the intersection of a non-empty collection of sets. Also \(B\) contains \(A\) since every set 
      in \(S\) does.
      
      Next suppose \(a_1, \ldots, a_{n_{i-1}} \in B\) and \((a_1, \ldots, a_{n_{i-1}}, a_{n_i}) \in R_i\). Since
      \(B\) is the intersection, all sets in \(S\) contain \(a_1, \ldots, a_{n_{i-1}}\). Additionally, since all
      sets in \(S\) are closed, they must also contain \(a_{n_i}\), implying \(B\) must as well, and \(B\) is
      closed under \(R_i\).
      
      Lastly \(B\) must be minimal since if a proper subset \(B'\) existed, it would be a member of \(S\), and
      thus it would include \(B\).
    \end{proof}
    
  \dbullet{1.6.6}
    The \(B\) in Theorem 1.6.5 is the "closure" of \(A\) under the relations \(\inflatedot{R}{m}\).
    
  \dbullet{1.7.1}
    An "alphabet" is a finite set of "symbols." A "string" over an alphabet is a finite sequence of symbols from the
    alphabet. The "empty string," denoted \(e\), is a string with no symbols. The set of all strings, including the
    empty string, over an alphabet \(\Sigma\) is denoted \(\Sigma^*\).
    
  \dbullet{1.7.2}
    The "concatentation" of strings \(x\) and \(y\), denoted \(x = y\) or \(xy\), is the string \(x\) followed by
    string \(y\). A string \(v\) is a "substring" of a string \(w\) if and only if there are strings \(x\) and \(y\)
    such that \(w = xvy\). If \(w = xv\), then \(v\) is a "suffix" of \(w\). If \(w = vy\), then \(v\) is a "prefix."
    For each string \(w\) and \(i \in \bbn\), \(w^0 = e\) and \(w^{i+1} = w^i \circ w\) for each \(i \geq 0\). The
    "reversal" of a string \(w\) is denoted \(w^R\).
    
  \dbullet{1.7.3}
    Any subset of \(\Sigma^*\) over alphabet \(\Sigma\) is a "language." When a particular alphabet \(\Sigma\) is
    understood from context, we write \(\bar{A}\), the "complement" of \(A\), instead of \(\Sigma^*-A\). If \(L_1\)
    and \(L_2\) are languages, their "concatenation," denoted \(L_1 \circ L_2\) or \(L_1L_2\) is \(\{w \in \Sigma^*:
    w = x \circ y\text{ for some }x \in L_1\text{ and }y \in L_2\}\).
    
  \dbullet{1.7.4}
    The "Kleene stra" of a language \(L\), denoted \(L^*\), is the set of all strings obtained by concatenating zero
    or more strings from \(L\). The "closure of \(L\) under concatenation," denoted \(L^+\), is the language \(LL^*\).
    
  \dbullet{1.8.1}
    The "regular expressions" over an alphabet \(\Sigma^*\) are all strings over the alphabet \(\Sigma \cup \{\pmb{(},
    \pmb{)}, \pmb{\emptyset}, \pmb{\cup}, \pmb{*}\}\) that can be obtained as follows:
    \begin{enumerate}[i.]
      \item \(\pmb{\emptyset}\) and each member of \(\Sigma\) is a regular expression.
      \item If \(\alpha\) and \(\beta\) are regular expressions, then so is \((\alpha\beta)\).
      \item If \(\alpha\) and \(\beta\) are regular expressions, then so is \((\alpha\pmb{\cup}\beta)\).
      \item If \(\alpha\) is a regular expression, then so is \(\alpha^{\pmb{*}}\).
      \item Nothing is a regular expression unless it follows from (\romannumeral 1) to (\romannumeral 4).
    \end{enumerate}
      
  \dbullet{1.8.2}
    The relation between regular expressions and the languages they represent is established by the function
    "\(\mathcal{L}\)" defined as follows:
    \begin{enumerate}[i.]
      \item 
        \(\mathcal{L}(\pmb{\emptyset}) = \emptyset\) and \(\mathcal{L}(a)=\{a\}\) for each \(a \in \Sigma\).
      \item 
        If \(\alpha\) and \(\beta\) are regular expressions, then 
        \(\mathcal{L}(\pmb{(}\alpha\beta\pmb{)}) = \mathcal{L}(\alpha)\mathcal{L}(\beta)\).
      \item
        If \(\alpha\) and \(\beta\) are regular expressions, then
        \(\mathcal{L}(\pmb{(}\alpha\pmb{\cup}\beta\pmb{)}) = \mathcal{L}(\alpha)\cup\mathcal{L}(\beta)\).
      \item
        If \(\alpha\) is a regular expressoin, then \(\mathcal{L}(\alpha^{\pmb{*}}) = \mathcal{L}(\alpha)^*\).
    \end{enumerate}

  \dbullet{1.8.3}
    The class of "regular languages" over an alphabet \(\Sigma\) is defined to consist of all languages \(L\) such
    that \(L = \mathcal{L}(\alpha)\) for some regular expression \(\alpha\) over \(\Sigma\).
    
  \dbullet{1.8.4}
    An algorithm that is specifically designed, for some language \(L\), to answer questions of the form
    "Is string \(w\) a member of \(L\)?" is a "language recognition device."
    
  \dbullet{1.8.5}
    A regular expression (and other finite language specifications) describing how a general specimen in a language is
    produced is a "language generator."

\end{outline}

\end{document}
