\documentclass[a4paper,11pt]{article}
\usepackage[a4paper, margin=20mm]{geometry}
\usepackage[T1]{fontenc}
\usepackage[utf8]{inputenc}
\usepackage{mathfunc}

% Header
% ===========================

\title{Homomorphisms}
\author{John B. Fraleigh, A First Course In Abstract Algebra (Chapter 13)}
\date{May 27, 2015}


% Document
% ===========================

\begin{document}
\maketitle
\pagenumbering{gobble}

\begin{outline}

  \dbullet{13.1}
    A map \(\phi\) of a group \(G\) into a group \(G'\) is a "homomorphism" if the
    homomorphism property \(\phi(ab) = \phi(a)\phi(b)\) holds for all \(a, b \in G\).
    
  \dbullet{13.2}
    Let \(\phi\) be a mapping of a set \(X\) into a set \(Y\), and let \(A \subseteq X\)
    and \(B \subseteq Y\). The "image \(\phi[A]\) of \(A\) in \(Y\) under \(\phi\)" is \(\{\phi(a) : a \in A\}\).
    The set \(\phi[X]\) is the "range of \(\phi\)." The "inverse image \(\phi^{-1}[B]\) of \(B\) in \(X\)" is
    \(\{x \in X : \phi(x) \in B\}\).
    
  \tbullet{13.3}
    Let \(\phi\) be a homomorphism of a group \(G\) into a group \(G'\).
    \begin{enumerate}[i.]
      \item If \(e\) is the identity in \(G\), the \(\phi(e)\) is the identity \(e'\) in \(G'\).
      \item If \(a \in G\), then \(\phi(a^{-1}) = \phi(a)^{-1}\).
      \item If \(H\) is a subgroup of \(G\), then \(\phi[H]\) is a subgroup of \(G'\).
      \item If \(K'\) is a subgroup of \(G'\), then \(\phi^{-1}[K']\) is a subgroup of \(G\).
    \end{enumerate}
    
    \begin{proof}
      Let \(\phi\) be a homomorphism of \(G\) into \(G'\).
      \begin{enumerate}[i.]
        \item 
          Then \(\phi(a) = \phi(ae) = \phi(a)\phi(e) \Rightarrow \phi(a)^{-1}\phi(a)\phi(e) \Rightarrow e' = \phi(e)\).
        \item 
          Next \(e' = \phi(e) = \phi(aa^{-1}) = \phi(a)\phi(a^{-1}) \Rightarrow \phi(a^{-1}) = \phi(a)^{-1}\).
        \item 
          Now, let \(H\) be a subgroup of \(G\) and let \(\phi(a), \phi(b) \in \phi[H]\). Then \(\phi(a)\phi(b) = 
          \phi(ab)\) so \(\phi(a)\phi(b) \in \phi[H]\) and \(\phi[H]\) is closed. Since \(e' = \phi(e)\) and 
          \(\phi(a^{-1}) = \phi(a)^{-1}\), \(\phi[H]\) is a subgroup of \(G'\).
        \item 
          Lastly, let \(K'\) be a subgroup of \(G'\) and suppose \(a, b \in \phi^{-1}[K']\). Then
          \(\phi(a)\phi(b) \in K'\) since \(K'\) is a subgroup. The equation \(\phi(ab) = \phi(a)\phi(b)\) shows that
          \(ab \in \phi^{-1}[K']\). Thus \(\phi^{-1}[K']\) is closed under the binary operation in G. \(K'\) must contain
          the identity element \(e' = \phi(e)\) so \(e \in \phi^{-1}[K']\). If \(a \in \phi^{-1}[K']\), then
          \(\phi(a) \in K'\), so \(\phi(a)^{-1} \in K'\). But note \(\phi(a)^{-1} = \phi(a^{-1})\), so \(a^{-1} \in
          \phi^{-1}[K']\). Hence \(\phi^{-1}[K']\) is a subgroup.
      \end{enumerate}
    \end{proof}

  \dbullet{13.4}
    Let \(\phi: G \rightarrow G'\) be a homomorphism of groups. The subgroup \(\phi^{-1}[\{e\}]\) is
    the "kernel of \(\phi\)," denoted by \(Ker(\phi)\).
    
  \tbullet{13.5}
    Let \(\phi: G \rightarrow G'\) be a group homomorphism, and let \(H = Ker(\phi)\). Let \(a \in G\).
    Then the set \(\phi^{-1}[\{\phi(a)\}] = \{x \in G : \phi(x) = \phi(a)\}\) is the left coset \(aH\) of \(H\), and is
    also the right coset \(Ha\) of \(H\). 
    
    \begin{proof}
      We want to show \(\{x \in G : \phi(x) = \phi(a)\} = aH\). Suppose \(\phi(x) = \phi(a)\). Then \(\phi(a)^{-1}\phi(x)
      = e'\), where \(e'\) is the identity in \(G'\). This implies \(\phi(a^{-1})\phi(x) = e'\) and, in turn, \(\phi(a^{-1}x)
      = e'\). Thus \(a^{-1}x \in H = Ker(\phi)\), so \(a^{-1}x = h\) for some \(h \in H\), and \(x = ah \in aH\). Therefore
      \(\{x \in G : \phi(x) = \phi(a)\} \subseteq aH\). Now let \(y \in aH\) so \(y = ah\) for some \(h \in H\). Then
      \(\phi(y) = \phi(ah) = \phi(a)\phi(h) = \phi(a)e' = \phi(a)\). So \(y \in \{x \in G : \phi(x) = \phi(a)\}\). The
      right cosets \(Ha\) are proved similarly.
    \end{proof}
    
  \cbullet{13.6}
    A group homomorphism \(\phi: G \rightarrow G'\) is a one-to-one map if and only if \(Ker(\phi) = \{e\}\).
    
    \begin{proof}
      \forward 
        Suppose \(\phi\) is one-to-one. We note \(\phi(e) = e'\) and we see, since \(\phi\) is injective, that
        \(e\) is the only element mapped to \(e'\) so \(Ker(\phi) = \{e\}\).
        
      \backward 
        If \(Ker(\phi) = \{e\}\), then for every \(a \in G\), the elements mapped into \(\phi(a)\) are precisely
        the elements of the left coset \(a\{e\} = \{a\}\), showing \(\phi\) is one-to-one.    
    \end{proof}
    
  \dbullet{13.7}
    A subgroup \(H\) of a group \(G\) is "normal" if its left and right cosets coincide, that is, if \(gH = Hg\) 
    for all \(g \in G\).
    
  \cbullet{13.8}
    If \(\phi: G \rightarrow G'\) is a group homomorphism, then \(Ker(\phi)\) is a normal subgroup of \(G\).

\end{outline}

\end{document}
