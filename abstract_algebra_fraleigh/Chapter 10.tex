\documentclass[a4paper,11pt]{article}
\usepackage[a4paper, margin=20mm]{geometry}
\usepackage[T1]{fontenc}
\usepackage[utf8]{inputenc}
\usepackage{mathfunc}


% Macros
% ===========================

\newcommand{\lowtilde}[1] {
  \textasciitilde{}\(_{#1}\)
}


% Header
% ===========================

\title{Cosets and the Theorem of Lagrange}
\author{John B. Fraleigh, A First Course In Abstract Algebra (Chapter 10)}
\date{May 27\textsuperscript{th}, 2015}


% Document
% ===========================

\begin{document}
\maketitle
\pagenumbering{gobble}

\begin{outline}

  \tbullet{10.1}
    Let \(H\) be a subgroup of \(G\). Let the relation\lowtilde{L}be defined on \(G\) 
    by \(a\)\lowtilde{L}\(b\) if and only if \(a^{-1}b \in H\). Similarly, let\lowtilde{R}be defined 
    by \(a\)\lowtilde{R}\(b\) if and only if \(ab^{-1} \in H\). Then\lowtilde{L}and\lowtilde{R}are 
    equivalence relations.
    
    \begin{proof}
      We will prove\lowtilde{R}is an equivalence relation and omit the symmetric proof for\lowtilde{R}.
      
      \textbf{Reflexive}: Let \(a \in G\). The \(a^{-1}a = e\) and \(e \in H\) since \(H\) is a subgroup.
      
      \textbf{Symmetric}: Suppose \(a\)\lowtilde{L}\(b\). Then \(a^{-1}b \in H\). Since \(H\) is a subgroup,
      \((a^{-1}b)^{-1} \in H\) implies \(b^{-1}a \in H\) and thus \(b\)\lowtilde{L}\(a\).
      
      \textbf{Transitive}: Let \(a\)\lowtilde{L}\(b\) and \(b\)\lowtilde{L}\(c\). The \(a^{-1}b \in H\) 
      and \(b^{-1}c \in H\). Since H is a subgroup, \((a^{-1}b)(b^{-1}c) = a^{-1}c \in H\), so 
      \(a\)\lowtilde{L}\(c\).
    \end{proof}
    
  \dbullet{10.2}
    Let \(H\) be a subgroup of a group \(G\). The subset \(aH = \{ah : h \in H\}\) of \(G\)
    is the "left coset" of \(H\) containing \(a\), while the subset \(Ha = \{ha : h \in H\}\) is the "right
    coset" of \(H\) containing \(a\).
    
  \tbullet{10.3 (Theorem of Lagrange)}
    Let \(H\) be a subgroup of a finite group \(G\). The the order of \(H\) is a divisor of the order of \(G\).
    
    \begin{proof}
      Let \(n\) be the order of \(G\), and let \(H\) have order \(m\). Note every coset of \(H\) must have
      the same order as \(H\). Let \(r\) be the number of cells in the partition of \(G\) into left cosets of
      \(H\). Then \(n = rm\), so \(m\) is indeed a divisor of \(n\).
    \end{proof}
    
  \cbullet{10.4}
    Every group of prime order is cyclic.
    
    \begin{proof}
      Let \(G\) be of prime order \(p\), and let \(a\) be an element of \(G\) different from the identity. Then
      the cyclic subgroup \(\langle a \rangle\) of \(G\) has at least two elements, \(a\) and \(e\). We note the
      order \(m \geq 2\) of \(\langle a \rangle\) must divide \(p\) by the Theorem of Lagrange so \(m = p\), 
      \(\langle a \rangle = G\), and \(G\) is cyclic.
    \end{proof}
    
  \tbullet{10.5}
    The order of an element of a finite group divides the order of the group.
    
  \dbullet{10.6}
    Let \(H\) be a subgroup of a group \(G\). The number of left cosets of \(H\) in \(G\)
    is the "index \([G:H]\) of \(H\) in \(G\)."
    
  \tbullet{10.7}
    Suppose \(H\) and \(K\) are subgroups of a group \(G\) such that \(K \leq H \leq G\), and suppose \([H:K]\) 
    and \([G:H]\) are both finite. Then \([G:K]\) is finite and \([G:K] = [G:H][H:K]\).

\end{outline}

\end{document}
