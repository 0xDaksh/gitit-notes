\documentclass[a4paper,11pt]{article}
\usepackage[a4paper, margin=20mm]{geometry}
\usepackage[T1]{fontenc}
\usepackage[utf8]{inputenc}
\usepackage{mathfunc}

% Header
% ===========================

\title{Applications of \(G\)-sets to Counting}
\author{John B. Fraleigh, A First Course In Abstract Algebra (Chapter 17)}
\date{May 27\textsuperscript{th}, 2015}


% Document
% ===========================

\begin{document}
\maketitle
\pagenumbering{gobble}

\begin{outline}

  \tbullet{17.1 (Burnside's Formula)}
    Let \(G\) be a finite group and \(X\) a finite \(G\)-set. If \(r\) is the number of orbits in \(X\) 
    under \(G\), then \(r \cdot |G| = \sum_{g \in G}|X_g|\).
    
    \begin{proof}
      We consider all pairs \((g, x)\) where \(gx = x\), and let \(N\) be the number of such pairs. For each
      \(g \in G\) there are \(|X_g|\) pairs having \(g\) as first member. Thus \(N = \sum_{g \in G}|X_g|\).
      On the other hand, for each \(x \in X\) there are \(|Gx|\) pairs having \(x\) as second member. Thus we also
      have \(N = \sum_{x \in X} |G_x|\). We note \(|Gx| = [G:G_x] = \frac{|G|}{|G_x|}\) so \(|G_x| = \frac{|G|}{|Gx|}\).
      Then:
      
      \begin{equation}
        \label{eqn1}
        N = \sum_{x \in X}(\frac{|G|}{|Gx|}) = |G|\sum_{x \in X} (\frac{1}{|Gx|}).
      \end{equation}
      
      Now \(\frac{1}{|Gx|}\) has the same value for all \(x\) in the same orbit, so for any orbit \(O\):
      
      \begin{equation}
        \label{eqn2}
        \sum_{x \in O} \frac{1}{|Gx|} = \sum_{X \in O} \frac{1}{|O|} = |O| \cdot \frac{1}{|O|} = 1
      \end{equation}
      
      Therefore substituting \eqref{eqn1} into \eqref{eqn2}, \(N = |G| \cdot r\).
    \end{proof}

  \cbullet{17.2}
    If \(G\) is a finite group and \(X\) is a finite \(G\)-set, then the number of orbits is 
    \(\frac{1}{|G|}\sum_{g \in G}|X_g|\).

\end{outline}

\end{document}
